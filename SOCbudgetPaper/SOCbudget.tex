%% Copernicus Publications Manuscript Preparation Template for LaTeX Submissions
%% ---------------------------------
%% This template should be used for copernicus.cls
%% The class file and some style files are bundled in the Copernicus Latex Package, which can be downloaded from the different journal webpages.
%% For further assistance please contact Copernicus Publications at: production@copernicus.org
%% https://publications.copernicus.org/for_authors/manuscript_preparation.html

%% copernicus_rticles_template (flag for rticles template detection - do not remove!)

%% Please use the following documentclass and journal abbreviations for discussion papers and final revised papers.

%% 2-column papers and discussion papers
\documentclass[gc, manuscript]{copernicus}



%% Journal abbreviations (please use the same for discussion papers and final revised papers)


% Advances in Geosciences (adgeo)
% Advances in Radio Science (ars)
% Advances in Science and Research (asr)
% Advances in Statistical Climatology, Meteorology and Oceanography (ascmo)
% Annales Geophysicae (angeo)
% Archives Animal Breeding (aab)
% ASTRA Proceedings (ap)
% Atmospheric Chemistry and Physics (acp)
% Atmospheric Measurement Techniques (amt)
% Biogeosciences (bg)
% Climate of the Past (cp)
% DEUQUA Special Publications (deuquasp)
% Drinking Water Engineering and Science (dwes)
% Earth Surface Dynamics (esurf)
% Earth System Dynamics (esd)
% Earth System Science Data (essd)
% E&G Quaternary Science Journal (egqsj)
% Fossil Record (fr)
% Geochronology (gchron)
% Geographica Helvetica (gh)
% Geoscience Communication (gc)
% Geoscientific Instrumentation, Methods and Data Systems (gi)
% Geoscientific Model Development (gmd)
% History of Geo- and Space Sciences (hgss)
% Hydrology and Earth System Sciences (hess)
% Journal of Micropalaeontology (jm)
% Journal of Sensors and Sensor Systems (jsss)
% Mechanical Sciences (ms)
% Natural Hazards and Earth System Sciences (nhess)
% Nonlinear Processes in Geophysics (npg)
% Ocean Science (os)
% Primate Biology (pb)
% Proceedings of the International Association of Hydrological Sciences (piahs)
% Scientific Drilling (sd)
% SOIL (soil)
% Solid Earth (se)
% The Cryosphere (tc)
% Web Ecology (we)
% Wind Energy Science (wes)


%% \usepackage commands included in the copernicus.cls:
%\usepackage[german, english]{babel}
%\usepackage{tabularx}
%\usepackage{cancel}
%\usepackage{multirow}
%\usepackage{supertabular}
%\usepackage{algorithmic}
%\usepackage{algorithm}
%\usepackage{amsthm}
%\usepackage{float}
%\usepackage{subfig}
%\usepackage{rotating}


% The "Technical instructions for LaTex" by Copernicus require _not_ to insert any additional packages.
%
\usepackage{algorithmic}
\usepackage{algorithm}


\begin{document}

\title{Historical Soil Organic Carbon Budget}


\Author[1]{Kristine}{Karstens}
\Author[1]{Benjamin Leon}{Bodirsky}
\Author[1]{Alexander}{Popp}


\affil[1]{Potsdam-Institut of Climate Impacts Research, Potsdam, Germany}

%% The [] brackets identify the author with the corresponding affiliation. 1, 2, 3, etc. should be inserted.



\runningtitle{R Markdown Template for Copernicus}

\runningauthor{Nüst et al.}


\correspondence{Kristine\ Karstens\ (\href{mailto:kristine.karstenst@pik-potsdam.de}{\nolinkurl{kristine.karstenst@pik-potsdam.de}})}



\received{}
\pubdiscuss{} %% only important for two-stage journals
\revised{}
\accepted{}
\published{}

%% These dates will be inserted by Copernicus Publications during the typesetting process.


\firstpage{1}

\maketitle


\begin{abstract}
SOC one of larges c sinks on earth (3 times larger biospehre pool). Agricultural management leads to a depletion of soil organic crabon. However
this depletion of soil organic carbon (SOC) pools are so far not well represented in global assessments of historic carbon emissions. While SOC
models often represent well the biochemical processes that lead to the accumulation and decay of SOC, the management decisions driving these
biophysical processes are still little investigated.
Here we create a spatial explicit data set for crop residue and manure management on cropland based on global historic production (FAOSTAT) and
land-use (LUH2) data and combine it with the IPCC Tier 2 approach to create a half-degree resolution soil organic carbon budget on mineral soils.
We estimate that due to arable farming soils have lost over (?) GtOC of which (??) GtOC have been released within the period 1990-2010.
Tier 2 IPCC methodolgy estimates higher soil organic carbon losses than Tier 1 methods, which may origin from \ldots{} . We also find that SOC is very
sensity to management decision such as residue recycling indicating the nessessity to incorporated better management data in soil models.
\end{abstract}


\copyrightstatement{The author's copyright for this publication is transferred to institution/company.}


\newpage

\introduction

Introduction text goes here.
You can change the name of the section if neccessary using \texttt{\textbackslash{}introduction{[}modified\ heading{]}}.
\newpage

\hypertarget{method-50}{%
\section{Method (50)}\label{method-50}}

\hypertarget{sec:carbonbudget}{%
\subsection{Carbon Stocks following (new) Tier 2 method (50)}\label{sec:carbonbudget}}

We calculate annual land use type specific soil organic carbon stocks for cropland, pastures and natural vegetation on half-degree resolution for the period of 1965 to 2010 based on the following three steps: (1) Calculating the land use (sub-)type specific steady-states and decay rates for SOC stocks given the current biophysical, climatic and agronomic conditions, (2) accounting for land conversation effects by transferring SOC between land use types and (3) updating SOC stocks based on the previous stock, the steady-state and the decay rate.

\hypertarget{steady-state-soc-stocks-and-decay-rates}{%
\subsubsection{Steady-state SOC stocks and decay rates}\label{steady-state-soc-stocks-and-decay-rates}}

In a simple first order kinetic approach the steady-state soil organic carbon stocks \(SOC^{eq}\) are given by
\begin{equation}
SOC^{eq} =\frac{C^{\textrm{in}}}{k}
\label{eq:inoutflow}
\end{equation}
with \(C^{\textrm{in}}\) being carbon inputs to the soil and \(k\) denoting the soil organic carbon decay rate. We use for our calculations the steady-state method of the refinement of the IPCC guidelines vol.~4 (\citet{ipcc_2019_2019}) for mineral soils, which assume three soil carbon pools and entangled dynamics between them. Carbon inflow to each subpool (see \ref{sec:carboninputs}) and decay rates (see \ref{sec:tier2}) of each subpool are still the key components to determining steady-state SOC stocks.

\hypertarget{sec:carboninputs}{%
\subsubsection{Carbon Inputs to the Soil}\label{sec:carboninputs}}

We account for different carbon inputs sources depending on the land use type (see table \ref{tab:datasourceinputs}). Following the IPCC methodology carbon inputs are disaggregated into different structural components depending on their lignin and nitrogen content (see @ref(ipcc\_2019\_2019)). For each structural components the sum over all carbon inputs sources is allocated to the respective SOC sub pools. This implies that not only the amount of carbon, but also their structural composition is determining the effective inflow. Data sources for all considered carbon inputs as well as for lignin and nitrogen content can be found in table \ref{tab:datasourceinputs}.

 \begin{table*}[h]
 \caption{Type and data sources for carbon inputs to different land use types}
 \begin{tabular}{l l l l}
 \tophline
  Land use types   & source of carbon inputs & data source & nitrogen and lignin content \\
 \middlehline
 \multirow{3}{*}{Cropland} & residues & FAOSTAT, LPJmL4 [2, sec:residues] & default values given by [2]  \\
                            & dead below ground biomass of crops & FAOSTAT, LPJmL4 [2, sec:residues] & default values given by [2] \\
                            & manure & FAOSTAT, Isabelle [2, sec:manure] & default values given by [2] \\
                            \hline
 \multirow{2}{*}{Pasture}  & annual litterfall & LPJmL4 [3] & default values given by [2] \\ 
                            & manure  & FAOSTAT, Isabelle [2, sec:manure] & default values given by [2] \\
                            \hline
  Natural vegetation        & annual litterfall & LPJmL4 [4]& \begin{minipage}[t]{0.28\columnwidth}\raggedright\strut Nitrogen and lignin content of tree compartments used in CENTURY [4] \strut \end{minipage}\tabularnewline
 \bottomhline
 \end{tabular}
 \label{tab:datasourceinputs}
 \belowtable{}
 \end{table*}

\hypertarget{sec:tier2}{%
\subsubsection{Soil Organic Carbon decay (300)}\label{sec:tier2}}

Decay rates are influenced by climatic conditions, biophysical and biochemical soil properties as well as management factors that vary over time (t) and space (i). Following the steady-state method of the refinement of the IPCC guidelines vol.~4 (\citet{ipcc_2019_2019}) for mineral soils we consider temperature (temp), water (wat), sand fraction (sf) and tillage (till) effects to spatially disaggregate default global decay rates \(k_{sub}\) given by the IPCC via

\begin{equation}
k_{sub,t,i} = \quad k_{sub} \quad \cdot temp_{t,i} \quad \cdot wat_{t,i} \underbrace{\cdot sf_{t,i} \quad \cdot till_{t,i}.}_{\text{only included for some subpools}}
\label{eq:decayrates}
\end{equation}

For cropland we performed an assessment of tillage types and irrigation conditions, whereas on pastures and natural vegetation, we assume rainfed and non-tilled conditions. Data sources as well as considered effects for each land use types are shown in table \ref{tab:datasourcedecay}. To account for variations of decay rates within grid cells based on different tillage and irrigation regimes, average rates based on area shares are calculated.

 \begin{table*}[h]
 \caption{Type and data sources for carbon inputs to different land use types}
 \begin{tabular}{l l l l}
 \tophline
  Land use types   & type of decay driver & parameter use to represent driver & data source \\
 \middlehline
 \multirow{2}{*}{all} & Soil quality & Sand fraction of the first 0-30 cm &  [SoilGrids]  \\
                      \cline{2-4}
                      
                      & Mircobial activity & air temperature & [CRUp4.0] \\
                      \cline{2-4}
                      
                      & Water restriction & precipitation \& potential evapotranspiration & [CRUp4.0] \\
                      \cline{1-4}
\multirow{2}{*}{\begin{minipage}[t]{0.2\columnwidth}\raggedright\strut Cropland\\(additionally)\strut\end{minipage}} & Water restriction*  & irrigation  & [sec:irrigation] \\ 
                      \cline{2-4}
                      
                      & Soil disturbance & tillage & [sec:tillage] \\
 \bottomhline
 \end{tabular}
 \label{tab:datasourcedecay}
 \belowtable{}
 \end{table*}

\hypertarget{soc-transfer-between-land-use-types}{%
\subsubsection{SOC transfer between land use types}\label{soc-transfer-between-land-use-types}}

We calculate SOC stocks based on the area shares of land use types (lut) within the half-degree grid cells (i). If land is converted from one land use type into another, a respective share of the SOC stocks have to be reallocated as well. We account for land conversion at the beginning of each time step \(t\) by calculating a preliminary stock \(SOC_{lut,t*}\) via

\begin{equation}
SOC_{lut,t*} = SOC_{lut,t-1} - \frac{SOC_{lut,t-1}}{A_{lut,t-1}} \cdot  AR_{lut,t} + \frac{SOC_{!lut,t-1}}{A_{!lut,t-1}} \cdot  AE_{lut,t} \qquad \forall sub, i  
\label{eq:ctransfer}
\end{equation}

with \(A\) being the area, \(AR\) the area reduction and \(AE\) the area expansion for a given land use type \(lut\). Note that \(!lut\) denotes the (sum over) land use type(s), which decrease(s) in the specific time step \(t\). Data sources and methodology on land use states and changes are descripted in \ref{sec:landuse}.

\hypertarget{total-soc-stocks}{%
\subsubsection{Total SOC stocks}\label{total-soc-stocks}}

Carbon stocks \(SOC\) for each subpool (sub) converge towards the calculate steady-state stock \(SOC^{eq}\) for each land-use types (lut), each subpool (sub) and each annual time step (t) as represented in equation \eqref{eq:steadystate}.

\begin{equation}
SOC_{t} = SOC_{t-1} + (SOC^{eq}_{t} - SOC_{t-1}) \cdot k_{t} \cdot \Delta t \qquad \forall lut, sub, i.
\label{eq:steadystate}
\end{equation}

The global SOC stock for each time step can than be calculated via

\begin{equation}
SOC_{t} = \sum_{i} \underbrace{\sum_{lut} \overbrace{\sum_{sub} SOC_{lut, sub, t, i}.}^{\text{land use type specific SOC stock within a grid cell}}}_{\text{total SOC stock within a grid cell}}
\label{eq:totalstock}
\end{equation}

\newpage

\hypertarget{sec:tier1}{%
\subsection{Carbon Budget following Tier 1 (150)}\label{sec:tier1}}

Following the tier 1 approach of the IPCC guidelines vol.~4 (\citet{ipcc_2006_2006}), stocks are estimated via stock change factors given by the IPCC for the topsoil (0-30 cm) and based on measurements. The factors are differentiate between different crop and management systems reflecting different dynamics under changed in- and outflows without explicitly tracking these. They can be seen as conservative guesses and will be used to evaluate our modelling based results.

\hypertarget{sec:agrimanagement}{%
\subsection{Agricultural management (50)}\label{sec:agrimanagement}}

We combine data sets to estimate agricultural flows and management decisions on cropland.

\hypertarget{sec:landuse}{%
\subsubsection{Landuse and Landuse Change (150)}\label{sec:landuse}}

We use the Land-Use Harmonization 2 (LUH2, \citep{LUH2}) half-degree data for major land-use states. Cropland is subdivided using country scale FAO data (\citep{FAOSTAT}) as well as crop type specific half-degree data from LUH2v2 to disaggregate into our 19 crop types on half degree resolution (see appendix for table on crop types and their mapping to FAOSTAT and LUH types as well as for technical discription of the disaggregation method).
Landuse transitions are calculate as area differences, allowing any land use type to either expand or shrink in each time step. Within cropland we do not track area transitions, following the averaging approach as descripted in \ref{sec:tier2}/\ref{sec:carbonbudget}

\hypertarget{crop-production-and-residues-300}{%
\subsubsection{Crop Production and Residues (300)}\label{crop-production-and-residues-300}}

FAOSTAT production (\citep{FAOSTAT}) values are combined with feed estimations including residue intake (\citep{weindl}) and rule based demand shares. LPJmL yield and calculate land use pattern (see @ref(\#sec:landuse)) are used to scale down to half-degree.

\hypertarget{livestock-distribution-and-manure-excretion-300}{%
\subsubsection{Livestock Distribution and Manure Excretion (300)}\label{livestock-distribution-and-manure-excretion-300}}

Based on {[}Gridded Livestock of the world{]} we use rule based asumption to estimate livestock and manure distribution on the globe. Animal waste system shares are used as is {[}Bodirsky{]}.

\hypertarget{irrigation-100}{%
\subsubsection{Irrigation (100)}\label{irrigation-100}}

Simple growing period calculations together with irrigation shares of LUH2v2 are use to estimate water effects on decay rates.

\hypertarget{tillage-100}{%
\subsubsection{Tillage (100)}\label{tillage-100}}

Tillage data sets of {[}Vera, others{]} together with rules are used to drive tillage effect on decay rates.

\newpage

\hypertarget{results}{%
\section{Results}\label{results}}

\newpage

\hypertarget{discussion}{%
\section{Discussion}\label{discussion}}

Shortcommings:

\begin{itemize}
\item
  Carbon displacement via leaching and erosion is neglected in this study.
\item
  Non-net/Gross land use transitions are not tracked in this study.
  \newpage
\end{itemize}

\conclusions

The conclusion goes here.
You can modify the section name with \texttt{\textbackslash{}conclusions{[}modified\ heading\ if\ necessary{]}}.
\newpage




\codedataavailability{use this to add a statement when having data sets and software code available} %% use this section when having data sets and software code available



%%%%%%%%%%%%%%%%%%%%%%%%%%%%%%%%%%%%%%%%%%
%% optional

%%%%%%%%%%%%%%%%%%%%%%%%%%%%%%%%%%%%%%%%%%
\appendix
\section{Figures and tables in appendices}
\subsection{Option 1}

If you sorted all figures and tables into the sections of the text, please also sort the appendix figures and appendix tables into the respective appendix sections.
They will be correctly named automatically.

\subsection{Option 2}

If you put all figures after the reference list, please insert appendix tables and figures after the normal tables and figures.

\texttt{\textbackslash{}appendixfigures} needs to be added in front of appendix figures
\texttt{\textbackslash{}appendixtables} needs to be added in front of appendix tables

Please add \texttt{\textbackslash{}clearpage} between each table and/or figure. Further guidelines on figures and tables can be found below.
Regarding figures and tables in appendices, the following two options are possible depending on your general handling of figures and tables in the manuscript environment:
To rename them correctly to A1, A2, etc., please add the following commands in front of them:
\noappendix

%%%%%%%%%%%%%%%%%%%%%%%%%%%%%%%%%%%%%%%%%%
\authorcontribution{Karstens wrote code and paper build on work of Bodirsky. Bodirsky and Popp revised paper.} %% optional section

%%%%%%%%%%%%%%%%%%%%%%%%%%%%%%%%%%%%%%%%%%
\competinginterests{The authors declare no competing interests.} %% this section is mandatory even if you declare that no competing interests are present

%%%%%%%%%%%%%%%%%%%%%%%%%%%%%%%%%%%%%%%%%%
\disclaimer{We like Copernicus.} %% optional section

%%%%%%%%%%%%%%%%%%%%%%%%%%%%%%%%%%%%%%%%%%
\begin{acknowledgements}
Thanks to the rticles contributors!
\end{acknowledgements}

%% REFERENCES
%% DN: pre-configured to BibTeX for rticles

%% The reference list is compiled as follows:
%%
%% \begin{thebibliography}{}
%%
%% \bibitem[AUTHOR(YEAR)]{LABEL1}
%% REFERENCE 1
%%
%% \bibitem[AUTHOR(YEAR)]{LABEL2}
%% REFERENCE 2
%%
%% \end{thebibliography}

%% Since the Copernicus LaTeX package includes the BibTeX style file copernicus.bst,
%% authors experienced with BibTeX only have to include the following two lines:
%%
\bibliographystyle{copernicus}
\bibliography{SOCbudget.bib}
%%
%% URLs and DOIs can be entered in your BibTeX file as:
%%
%% URL = {http://www.xyz.org/~jones/idx_g.htm}
%% DOI = {10.5194/xyz}


%% LITERATURE CITATIONS
%%
%% command                        & example result
%% \citet{jones90}|               & Jones et al. (1990)
%% \citep{jones90}|               & (Jones et al., 1990)
%% \citep{jones90,jones93}|       & (Jones et al., 1990, 1993)
%% \citep[p.~32]{jones90}|        & (Jones et al., 1990, p.~32)
%% \citep[e.g.,][]{jones90}|      & (e.g., Jones et al., 1990)
%% \citep[e.g.,][p.~32]{jones90}| & (e.g., Jones et al., 1990, p.~32)
%% \citeauthor{jones90}|          & Jones et al.
%% \citeyear{jones90}|            & 1990

\end{document}
