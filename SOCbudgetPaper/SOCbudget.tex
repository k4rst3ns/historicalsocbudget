%% Copernicus Publications Manuscript Preparation Template for LaTeX Submissions
%% ---------------------------------
%% This template should be used for copernicus.cls
%% The class file and some style files are bundled in the Copernicus Latex Package, which can be downloaded from the different journal webpages.
%% For further assistance please contact Copernicus Publications at: production@copernicus.org
%% https://publications.copernicus.org/for_authors/manuscript_preparation.html

%% copernicus_rticles_template (flag for rticles template detection - do not remove!)

%% Please use the following documentclass and journal abbreviations for discussion papers and final revised papers.

%% 2-column papers and discussion papers
\documentclass[gc, manuscript]{copernicus}



%% Journal abbreviations (please use the same for discussion papers and final revised papers)


% Advances in Geosciences (adgeo)
% Advances in Radio Science (ars)
% Advances in Science and Research (asr)
% Advances in Statistical Climatology, Meteorology and Oceanography (ascmo)
% Annales Geophysicae (angeo)
% Archives Animal Breeding (aab)
% ASTRA Proceedings (ap)
% Atmospheric Chemistry and Physics (acp)
% Atmospheric Measurement Techniques (amt)
% Biogeosciences (bg)
% Climate of the Past (cp)
% DEUQUA Special Publications (deuquasp)
% Drinking Water Engineering and Science (dwes)
% Earth Surface Dynamics (esurf)
% Earth System Dynamics (esd)
% Earth System Science Data (essd)
% E&G Quaternary Science Journal (egqsj)
% European Journal of Mineralogy (ejm)
% Fossil Record (fr)
% Geochronology (gchron)
% Geographica Helvetica (gh)
% Geoscience Communication (gc)
% Geoscientific Instrumentation, Methods and Data Systems (gi)
% Geoscientific Model Development (gmd)
% History of Geo- and Space Sciences (hgss)
% Hydrology and Earth System Sciences (hess)
% Journal of Micropalaeontology (jm)
% Journal of Sensors and Sensor Systems (jsss)
% Magnetic Resonance (mr)
% Mechanical Sciences (ms)
% Natural Hazards and Earth System Sciences (nhess)
% Nonlinear Processes in Geophysics (npg)
% Ocean Science (os)
% Primate Biology (pb)
% Proceedings of the International Association of Hydrological Sciences (piahs)
% Scientific Drilling (sd)
% SOIL (soil)
% Solid Earth (se)
% The Cryosphere (tc)
% Weather and Climate Dynamics (wcd)
% Web Ecology (we)
% Wind Energy Science (wes)


%% \usepackage commands included in the copernicus.cls:
%\usepackage[german, english]{babel}
%\usepackage{tabularx}
%\usepackage{cancel}
%\usepackage{multirow}
%\usepackage{supertabular}
%\usepackage{algorithmic}
%\usepackage{algorithm}
%\usepackage{amsthm}
%\usepackage{float}
%\usepackage{subfig}
%\usepackage{rotating}


% The "Technical instructions for LaTex" by Copernicus require _not_ to insert any additional packages.
%
\usepackage{algorithmic}
\usepackage{algorithm}


\begin{document}

\title{Management induced changes of soil organic carbon on global
croplands}


\Author[1]{Kristine}{Karstens}
\Author[1]{Benjamin Leon}{Bodirsky}
\Author[2]{Your}{Name}
\Author[1]{Alexander}{Popp}


\affil[1]{Potsdam-Institut of Climate Impacts Research, Potsdam,
Germany}
\affil[2]{Your affiliation}

\runningtitle{R Markdown Template for Copernicus}

\runningauthor{Nüst et al.}


\correspondence{Kristine\ Karstens\ (kristine.karstens@pik-potsdam.de)}



\received{}
\pubdiscuss{} %% only important for two-stage journals
\revised{}
\accepted{}
\published{}

%% These dates will be inserted by Copernicus Publications during the typesetting process.


\firstpage{1}

\maketitle


\begin{abstract}
Soil organic carbon (SOC) is one of larges carbon stocks on Earth. It is
three times larger than the biospheric pool and more than twice as big
as the athmospheric pool, when looking into the first meter of the earth
soil profile. Human cropping activties led and still lead to a depletion
of SOC though, which are so far not well represented in global
assessments of historic carbon emissions. While SOC models often
represent well the biochemical processes that lead to the accumulation
and decay of SOC, the management decisions driving these biophysical
processes are still little investigated on the global scale. Here we
create a spatial explicit data set for agricutural management on
cropland, especially for crop residue and manure management, based on
global historic production (FAOSTAT), and land-use (LUH2) data; and
combine it with the IPCC Tier 2 approach to create a half degree
resolution SOC stock changes on mineral soils. We estimate that due to
arable farming soils have lost over 37 GtOC of which 4 GtOC have been
regained within the period 1975-2010. We show that, our results on
global scale based on Tier 2 IPCC methodolgy are in good agreement with
Tier 1 default assumptions. We also find that SOC is very sensitive to
management decision such as residue recycling indicating the nessessity
to incorporated better management data in soil models.
\end{abstract}


\copyrightstatement{The author's copyright for this publication is
transferred to institution/company.}


\newpage

\introduction

Soil Organic Carbon (SOC), the amount of organic carbon stored through
the earth's soil, is the largest terrestrial carbon pool, exceeding the
carbon in the atmospheric and vegetation pools multiple times (Batjes).
As such, even small changes in drivers of SOC may thus lead to
substantial shifts in earth carbon cycle and influence the amount of CO2
in the atmosphere (ref. permafrost melting). The specific amount of
carbon stored in the soil is, however, uncertain, with estimates ranging
from 1500 to 2400 GtC for the first meter of the soil profile (Batjes,
1996; more). The quality of SOC maps has markedly improved in recent
decades, along with better understanding of the factors driving the
global magnitude, distribution, and dynamics of SOC pools. Natural
properties like climatic, biophysical, and landscape characteristics
clearly play the most important roles to determine SOC variations over
space and time. Human intervention, including land cover change and land
management, has however added a further driver to SOC change, which
alters terrestrial carbon pools in much shorter time scales and is
likely one of the most dominant driver of SOC changes on managed land
today (cite). Recent studies identify the anthropogenic SOC debt for the
first meter of the soil profile at around 116 GtC (Sanderman et al.),
compared to previous estimates between 60-130 GtC (Lal, 2006). Other
studies have focused more closely on spatially disaggregation of SOC
changes via advanced digital soil mapping techniques (S-World;
Stoorvogel 2, 2017) or better representation of biogeochemical processes
within SOC dynamics (Hararuk, 20XX). While providing a rough estimate of
the order of magnitude of change, these studies lack a detailed
consideration of land management, especially of agricultural activities.
Field-scale models (cite Daycent, RothC, Ecosse, C-Tool) are able to
capture these land management impacts by using detailed information on
crop yield levels, fertilizer inputs and various other on-farm measures.
However, due to the lack of comprehensive global management data as
input to these models, scaling up to the global extent remains a complex
challenge. Our study combines an spatially explicit estimate of
agricultural management data on the global level with a 3-pool SOC model
parametrized for global croplands. This allows us to estimate SOC stock
change factors, as well as organic carbon flow dynamics within the
agricultural system. We thereby consider change in SOC caused by
historical land cover change as well as of different agricultural
management practices, including residue recycling, manure amendments,
irrigation, and tillage. We provide the first global, spatially explicit
SOC loss maps that considers detailed agricultural management. This
paper will first introduce in the methods section the basic concept of
SOC dynamics as applied in this study. We continue with a detailed
description of the global gridded management data used here, including
crop production levels, residue recycling rates, manure amendments, and
the adoption of irrigation and tillage practices. Lastly, we shortly
refer to the concept of stock change factors as outlined in the Tier 1
approach of the IPCC guidelines. In our results section we focus on the
SOC dynamics of global croplands by (1) analysing the spatially explicit
distribution and depletion of SOC from 1975 to 2010, (2) comparing
impacts of different management effects on SOC emissions and (3)
quantify global agricultural carbon flows and stocks to compare the
importance of various management aspects. Finally, we discuss and
compare our findings -- including their implications for SOC model
development -- and conclude with an outlook on the ability of SOC
management to mitigate climate change and contribute to negative
greenhouse gas emissions. \newpage

\section{Method}

We compile calculations as open-source R packages available at
github.com/pik-piam/mrcommons (management related functions) and
github.com/pik-piam/mrSOCbudget (soil dynamic related functions), which
are both based on the MADRaT package (package citation?), a framework
which aims to improve reproducibility and transparency in data
processing. In the following chapter we outine the most important
relationships and assumptions. Table @ref(append:subsection2mrfunctions)
provides further information on corresponding code within the R
packages.

\hypertarget{sec:carbonbudget}{%
\subsection{SOC Stocks and Stock changes following the Tier 2
steady-state method}\label{sec:carbonbudget}}

Following the tier 2 approach of the refinement of IPCC guidelines
vol.~4 (\citet{ipcc_2019_2019}; short Tier 2 steady-state method) , we
estimate soil organic carbon (SOC) stocks and their change over time for
cropland on half-degree resolution from 1975 to 2010. We assume the
current SOC state converges towards a steady state, which itself is
depending on biophysical, climatic and agronomic conditions. Therefore
we conduct the following three steps within each yearly timestep: (1) We
calculate annual land-use (sub-)type-specific steady states and decay
rates for SOC stocks, (2) we account for land conversion by transferring
SOC from and to an other land-use type representad as natural
vegetation, (3) we estimate SOC stocks and changes based on the stocks
of the previous time period, the steady state stocks and the decay rate.

\subsubsection{Steady-state SOC stocks and decay rates}

In a simple first order kinetic approach the steady-state soil organic
carbon stocks \(SOC^{eq}\) are given by \begin{equation}
SOC^{eq} =\frac{C^{\textrm{in}}}{k} \qquad\forall\quad i,t
(\#eq:inoutflow)
\end{equation} with \(C^{\textrm{in}}\) being carbon inputs to the soil,
\(k\) denoting the soil organic carbon decay rate; as well as \(i\)
representing grid cell indices and \(t\) years. We use for our
calculations the Tier 2 steady-state method, which assumes three soil
carbon sub-pools (active, slow and passive) and interactions between
them. Annual carbon inflow to each sub-pool and annual decay rates of
each sub-pool are the key components to determining steady-state SOC
stocks.

\textbf{Carbon Inputs to the Soil}

We account for different carbon input sources depending on the two
land-use types we distinguish: croplands and natural vegetated land as
representative for all other land use (see table
@ref(tab:datasourceinputs)). Carbon sources for cropland are recycled
crop residues, below ground biomass of crops (for both see
@ref(sec:residues)) and recycled manure (see @ref(sec:livstmanure)); for
natural vegetation litterfall \citep{schaphoff_lpjml4_2018} is the only
source of carbon inflow to the soil.

Following the IPCC carbon accounting methodology, carbon inputs are
disaggregated into metabolic and structural components depending on
their lignin and nitrogen content (see \citet{ipcc_2019_2019}). For each
component the sum of all carbon input sources is allocated to the
respective SOC sub-pools via transfer coefficients. This implies that
both the amount of carbon as well as its structural composition
determine the effective inflow. Data sources for all considered carbon
inputs as well as for lignin and nitrogen content can be found in table
@ref(tab:datasourceinputs).

 \begin{table*}[h]
 \caption{Type and data sources for carbon inputs to different land-use types }
 \begin{tabular}{l l l l}
 \tophline
  \textbf{land-use types}   & \textbf{source of carbon inputs} & \textbf{data source} & \textbf{nitrogen and lignin content} \\
 \middlehline
 \multirow{3}{*}{Cropland} & residues & FAOSTAT, LPJmL4 [2, sec:residues] & default values given by [2]  \\
                            & dead below ground biomass of crops & FAOSTAT, LPJmL4 [2, sec:residues] & default values given by [2] \\
                            & manure & FAOSTAT, Isabelle [2, sec:manure] & default values given by [2] \\
                            \hline
  Natural vegetation        & annual litterfall & LPJmL4 [4]& \begin{minipage}[t]{0.28\columnwidth}\raggedright\strut Nitrogen and lignin content of tree compartments used in CENTURY [4] \strut \end{minipage}\tabularnewline
 \bottomhline
 \end{tabular}
 \label{tab:datasourceinputs}
 \belowtable{}
 \end{table*}

\textbf{SOC decay}

The sub-pool specific decay rates are influenced by climatic conditions,
biophysical and biochemical soil properties as well as management
factors that all vary over time (t) and space (i). Following the Tier 2
steady-state method we consider temperature (temp), water (wat), sand
fraction (sf) and tillage (till) effects to account for spatial
variation of decay rates. Thus \(k_{sub}\) is given by

\begin{equation}
\begin{aligned}
& k_{active,t,i}  & = &~ k_{active}  ~ &\cdot~ temp_{t,i} ~ &\cdot~ wat_{t,i} ~ &\cdot~ till_{t,i} ~ & \cdot~ sf_{t,i}\\
& k_{slow,t,i}    & = &~ k_{slow}    ~ &\cdot~ temp_{t,i} ~ &\cdot~ wat_{t,i} ~ &\cdot~ till_{t,i} ~ &\\
& k_{passive,t,i} & = &~ k_{passive} ~ &\cdot~ temp_{t,i} ~ &\cdot~ wat_{t,i} ~ & ~ &
(\#eq:decayrates)
\end{aligned}
\end{equation}

For cropland we distinguish the effect of different tillage (see
@ref(sec:tillage)) and irrigation (see @ref(sec:irrigation)) practices
on decay rates, whereas for natural vegetation, we assume rainfed and
non-tilled conditions. Data sources as well as drivers considered for
each land-use types are shown in table @ref(tab:datasourcedecay). To
account for variations of decay rates within each grid cell due to
different tillage and irrigation regimes, average rates based on area
shares are calculated.

 \begin{table*}[h]
 \caption{Type and data sources for carbon inputs to different land-use types}
 \begin{tabular}{l l l l}
 \tophline
  land-use types   & type of decay driver & parameter use to represent driver & data source \\
 \middlehline
 \multirow{2}{*}{all} & Soil quality & Sand fraction of the first 0-30 cm &  [SoilGrids]  \\
                      \cline{2-4}
                      
                      & Mircobial activity & air temperature & [CRUp4.0] \\
                      \cline{2-4}
                      
                      & Water restriction & precipitation \& potential evapotranspiration & [CRUp4.0] \\
                      \cline{1-4}
\multirow{2}{*}{\begin{minipage}[t]{0.2\columnwidth}\raggedright\strut Cropland\\(additionally)\strut\end{minipage}} & Water restriction*  & irrigation  & [sec:irrigation] \\ 
                      \cline{2-4}
                      
                      & Soil disturbance & tillage & [sec:tillage] \\
 \bottomhline
 \end{tabular}
 \belowtable{}
 (\#tab:datasourcedecay)
 \end{table*}

\subsubsection{SOC transfer between land-use types}

We calculate SOC stocks based on the area shares of land-use types (lut)
within the half-degree grid cells (i). If land is converted from one
land-use type into others (!lut), the respective share of the SOC stocks
is reallocated. We account for land conversion at the beginning of each
time step \(t\) by calculating a preliminary stock \(SOC_{lut,t*}\) via

\begin{equation}
SOC_{lut,t*} = SOC_{lut,t-1} - \frac{SOC_{lut,t-1}}{A_{lut,t-1}} \cdot  AR_{lut,t} + \frac{SOC_{!lut,t-1}}{A_{!lut,t-1}} \cdot  AE_{lut,t} \qquad \forall\quad sub, i  
(\#eq:ctransfer)
\end{equation}

with \(A\) being the area, \(AR\) the area reduction and \(AE\) the area
expansion for a given land-use type \(lut\). Note that \(!lut\) denotes
the sum over all other land-use types, which decreases in the specific
time step \(t\). Data sources and methodology on land-use states and
changes are described in @ref(sec:landuse).

\subsubsection{Total SOC stocks and stock changes}

Carbon stocks \(SOC\) for each sub-pool (sub) converge towards the
calculated steady-state stock \(SOC^{eq}\) for each land-use type (lut),
each sub-pool (sub) and each annual time step (t) like

\begin{equation}
SOC_{t} = SOC_{t-1} + (SOC^{eq}_{t} - SOC_{t-1}) \cdot k_{t} \cdot 1\unit{a} \qquad \forall\quad\quad lut, sub, i.
(\#eq:SOCstate)
\end{equation}

Reformulating this equation, we obtain a massbalance equation as follows

\begin{equation}
SOC_{t} = SOC_{t-1} - \underbrace{SOC_{t-1} \cdot k_{t} \cdot 1\unit{a}}_{\text{outflow}} + \overbrace{SOC^{eq}_{t} \cdot k_{t} \cdot 1\unit{a}}^{\text{input (using equation (1))}}  \qquad \forall\quad lut, sub, i.
(\#eq:steadystate2budget)
\end{equation}

The global SOC stock for each time step can than be calculated via

\begin{equation}
SOC_{t} = \sum_{i} \underbrace{\sum_{lut} \overbrace{\sum_{sub} SOC_{lut, sub, t, i}.}^{\text{$SOC_{lut, t, i}$ -- land-use type specific SOC stock within cell}}}_{\text{$SOC_{t, i}$ -- total SOC stock within cell}}
(\#eq:totalstock)
\end{equation}

According to the IPCC guidelines SOC changes can be calculates as the
difference of two (aufeinander folgenden) years (see Eq. 5.0A in
\citep{ipcc_2019_2019}). This however will also include naturally
occured changes due to climatic variance over time. For our study we
will define the absolute and relative SOC changes in relation to a
potential natural state \(SOC_{pnv}\) under the same climatic conditions
at time \(t\) in grid cell \(i\), that is based on the natural
vegetation SOC calculations as defined above without accounting for land
conversion from cropland at any time. The absolute changes
\(\Delta SOC\) and relative changes \(F^{SCF}\) a thus given by

\begin{equation}
\Delta SOC_{t, i} = SOC_{t, i} - SOC_{pnv, t, i}\quad, \qquad \quad  F^{SCF}_{t, i} = \frac{SOC_{t, i}}{SOC_{pnv, t, i}}.
(\#eq:totalstock)
\end{equation}

Note that the absolution changes \(\Delta SOC\) can be also interpreted
as the SOC gap due to human cropping activits; whereas relative changes
\(F^{SCF}\) can be denoted as stock change factors as defined within the
IPCC guidelines of 2006 \citep{ipcc_2006_2006}. Note that the SOC gap is
to the negated cummulative SOC emission also refered to as SOC debt
(cite Sanderman) of mankind.

\subsubsection{Initialisation of SOC pools}

To initialize all SOC sub-pools we assume that cropped land as well as
natural vegetation are in a steady state for the specific configuration
present within the initialization year 1961. We assume after a spin-up
period of 15 years reliable results start from 1975 and improve over
time, since dependency on the initial conditions will decrease.

\newpage

\hypertarget{sec:agrimanagement}{%
\subsection{Agricultural management data on 0.5 degree
resolution}\label{sec:agrimanagement}}

We compile country-specific FAO production and cropland statistics
\citep{faostat_faostat_2016} to a harmonized and consistent data set.
The data is prepared in 5 year time steps from 1965 to 2010, which,
together with the spin up phase, restricts our analysis to time span
from 1975 to 2010. For all the following data, if not declared
differently, we interpolate values linearly between the time steps and
hold them constant before 1965.

\hypertarget{sec:landuse}{%
\subsubsection{Landuse and Landuse Change}\label{sec:landuse}}

Land-use patterns are based on the Land-Use Harmonization 2
\citep{Hurtt2020} data set, which we aggregate from quarter degree to
half degree resolution. We disaggregate the five different cropland
subcategories (c3ann, c3per, c4ann, c4per, c3nfx) of LUH2 into our 17
crop groups, applying the relative shares for each grid cell based on
the country and year specific area harvested shares of FAOSTAT data
\citep{faostat_faostat_2016} (see @ref(append:Tableluh2fao2mag) for more
details on the crop group mapping). Land-use transitions are calculated
as net area differences of the land-use data on half-degree.

\hypertarget{sec:residues}{%
\subsubsection{Crop and Crop Residues Production}\label{sec:residues}}

\textbf{Crop Production}

Using half-degree yield data from LPJmL \citep{schaphoff_lpjml4_2018} as
well as half-degree cropland patterns (see @ref(sec:landuse)) we compile
crop group specific half-degree production patterns. We calibrate
cellular yields with a country-level calibration factor for each crop
group to meet historical FAOSTAT production
\citep{faostat_faostat_2016}. Note that by using physical cropland areas
we account for multiple crop harvest events as well as for fallows.

\textbf{Crop Residue Production}

Crop residue production and management is based on a revised methodology
of \citep{bodirsky_n2o_2012} and key aspects are explained here given
its central role for soil carbon modelling. Starting from harvested crop
production (\(CP\)) estimates and their respective harvested crop area
(\(CA\)), we estimate above-ground (\(AGR\)) and below-ground (\(BGR\))
residual biomass using crop group (\(cg\)) specific harvest indices
(\(HI\)) and root:shoot ratios (\(RS\)) as follows

\begin{equation}
\begin{aligned}
AGR & = CP \cdot HI_{\textrm{slope}} + CA \cdot HI_{\textrm{intercept}}\qquad & \textrm{and} \\
BGR & = (CP + AGR) \cdot RS \qquad                                            & \forall\quad cg, i, t.
(\#eq:resbiomass)
\end{aligned}
\end{equation}

Following the IPCC guidelines, we split the harvest index into a yield
and an area dependend fraction \citep{ipcc_2006_2006}. Note that
deviating from \citep{bodirsky_n2o_2012} we use harvested instead of
physical crop area to account for increased residue biomass due to
multiple cropping and decreased amounts on fallow land. We assume that
all BGR are recycled to the soil, whereas AGR can be burned or harvested
for other purposes such as feeding animals \citep{weindl}, fuel or for
material use.

\textbf{Burned Residues}

A fixed share of the AGR is assumed to be burned on field depending on
the per-capita income of the country. Following \citep{smil1999}) we
assume 25\% burn share for low-income countries according to worldbank
definitions (\(<\,1000\,\tfrac{USD}{yr\,cap}\)), 15\% for high-income
(\(>\,10000\,\tfrac{USD}{yr\,cap}\)) and linearly interpolate shares for
all middle-income countries depending on their per-capita income.
Depending on the crop group 80--90\% of the residue carbon burned on the
fields are lost within the combustion process \citep{ipcc_2006_2006}.

\textbf{Residue Usage}

We compile out of our 17 crop groups, three residue groups (straw,
high-lignin and low-lignin residues) with additional demand for other
purposes and one residue with no double use (see
@ref(append:Tablekcr2kres)). Residue feed demand for five different
livestock groups is based on country- and residue-group-specific feed
baskets (see \citep{weindl}), taking available AGR biomass as well as
livestock productivity into account. We estimate, for low-income
countries, a material-use share for straw residues of 5\% and a
fuel-share of 10\% for all used residues groups in low-income countries.
For high-income countries, no withdrawal for material or fuel use is
assumend, and middle-income countries use shares are linearly
interpolated based on per-capita income. The remaining AGR as well as
all BGR are assumend to be recycled to the soil. We limit high recycling
shares at \(10\tfrac{\unit{tC}}{\unit{ha}}\) to correct for outliers.

\textbf{Dry Matter to Carbon Transformation}

To transform dry matter estimates into carbon, we compiled crop-group
and plant part specific carbon to dry matter (c:dm) ratios (see
@ref(append:Tablec2dm)).

\hypertarget{sec:livstmanure}{%
\subsubsection{Livestock Distribution and Manure
Excretion}\label{sec:livstmanure}}

We assume that manure is applied at its excretion place, leaving the
livestock distribution the driving factor for the spatial pattern of
manuring.

\textbf{Livestock Distribution} To disaggregate country level FAOSTAT
livestock production values to the half-degree scale, we use the
following rule-based assumptions, drawing from the approach of
\citep{robinson_mapping_2014}, and uses feed basket assumptions based on
a revised methodology from \citep{weindl}. We differentiate between
ruminant and monogastric systems, as well as extensive and intensive
systems. For ruminants, we assume that livestock is located where the
production of feed occurs. We distingush between grazed pasture, which
is converted into livestock products from extensive systems; and all
other (crop-based) feeds, which we consider to be consumend in intensive
systems. For poultry, egg and monogastric meat production we use the
per-capita income of the country to divide between intensive and
extensive production systems. For low-income countries, we assume
extensive production systems. We locate them according to the share of
built-up areas based on the idea that these animals are held in
subsistence or small-holder farming systems with a high labour per
animal ratio. Intensive production associated with higher income
countries, is distributed within a country using again share of
crop-based feed production, assuming that feed availability is the most
driving factor for livestock location.

\textbf{Manure Excretion, Storage and Recycling} Manure production and
management is based on a revised methodology of
\citep{bodirsky_n2o_2012} and is presented here due to its central role
in soil carbon modelling. Based on the gridded livestock distribution we
calculate excretions by estimating the nitrogen balance of the livestock
system on the basis of comprehensive livestock feed baskets
\citep{weindl}, assuming that all nitrogen in protein feed intake, minus
the nitrogen in the slaughter mass, is excreted. Carbon in excreted
manure is estimated by applying fixed C:N ratios (given by
\citep{ipcc_2019_2019}). Depending on the feed system we assume manure
to be handled in four different ways: All manure orginated from pasture
feed intake is excreted directly to pastures and rangelands (pasture
grazing), deducting manure collected as fuel. Manure fuel shares are
estimated using IPCC default values \citep{ippc_2006_2006}. Whereas for
low-income countries, we adopt a share of 25\% of crop residues in feed
intake directly consumend and excreted on crop fields (stubble grazing),
we do not consider any stubble grazing in high-income countries;
middle-income countries see linearly interpolated shares depending on
their per-capita income. For all other feed items we assume the manure
to be stored in animal waste management systems associated with
livestock housing. To estimate the carbon actually recycled to the soil,
we account for carbon losses during storage and recycling shares in
different animal waste management and grazing systems. Whereas we assume
no losses for pasture and stubble grazing, we consider that the manure
collected as fuel is not recycled. For manure stored in different animal
waste management systems we compiled carbon loss rates partly depending
on the nitrogen loss rates as specified in \citep{bodirsky_n2o_2012}
(see @ref(append:TableclossAWMS)). We limit high application shares at
\(10\tfrac{\unit{tC}}{\unit{ha}}\) to correct for outliers.

\hypertarget{sec:irrigation}{%
\subsubsection{Irrigation}\label{sec:irrigation}}

We use irrigation area shares to modify the water effect on
decomposition by weighting the irrigated and rainfed water factors based
on these shares. The LUH2v2 data set provides irrigated fractions for
their cropland subcategories. We apply aggregated area shares, leading
to the same impact of irrigation on all of our crop groups. Moreover we
assume the irrigation effect to be present for all 12 months of a year
in grid cells, that has been marked as irrigated.

\hypertarget{sec:tillage}{%
\subsubsection{Tillage}\label{sec:tillage}}

To account for the distribution of tillage to the three different
tillage types specfied by the IPCC - full tillage, reduced tillage and
no tillage -, we assume that all natural land and pastures are not
tilled, whereas as default annual crops are under full and perennials
under reduced tillage. Furthermore we assume no tillage in cropland
cells specified as no tillage cell based on the historic global gridded
tillage data set from Prowollik \citep{porwollik_generating_2018}.
Porwollik data set is extended by to the period of 1974-2010 by
combining country-level data on conservation agriculture (ca) area
values from AQUASTAT (FAO, 2016) and LUH2 crop areas together with the
methodolgy of Porwollik to identify potential no tillage cells.

\newpage

\hypertarget{sec:tier1}{%
\subsection{SOC Stocks and Stock changes following Tier
1}\label{sec:tier1}}

Additionally to the tier 2 approach of the refinement of IPCC guidelines
vol.~4 \citep{ipcc_2019_2019} and the detailed analysis of management
data coming with it, SOC changes can be estimated using the IPCC tier 1
approach of IPCC guidelines vol.~4 (\citet{ipcc_2006_2006},
\citet{ipcc_2019_2019}). Here, stocks are calculated via stock change
factors (F\^{}\{SCF\}) given by the IPCC for the topsoil (0-30 cm) and
based on observational data. \(F^{SCF}\) are differentiated by different
crop, management and input systems (here summarized under \(m\))
reflecting different dynamics under changed in- and outflows without
explicitly tracking these. Moreover \(F^{SCF}\) vary for different
climatic zones (\(c\)) specified by the IPCC (see
@ref(append:climatemap)). The actual SOC stocks as thus calulated based
on a given reference SOC stock by

\begin{equation}
SOC^{\text{target}}_{i} = \sum_{c,m} T_{c,i} \cdot SOC^{\text{ref}} \cdot F^{SCF}_{c,m} \qquad\forall\quad t,
(\#eq:tier1)
\end{equation}

with \(T_{c,i}\) being the translation matrix for grid cells \(i\) into
corresponding climate zones \(c\). For this analysis we use the default
\(F^{SCF}\) from the Tier 1 method of \citep{ipcc_2006_2006}, and
\citep{ipcc_2019_2019} as a comparison and constistency check for our
more detailed budget approach. \newpage

\section{Results}

We present simulation results of our SOC budget focusing on cropland
areas for the year 2010 as well as global trends of SOC stock changes
for the period from 1975 to 2010.

\subsection{SOC distribution and depletion}

\begin{figure}[H]
\includegraphics[width=18cm]{../ResultNotebooks/Output/Images/4panelfigure} \caption{(a): Distribution of total global SOC stocks on cropland shows high carbon stocks in high yielding areas. (b): The SOC debt is decreasing over time, meaning net SOC gains on global croplands over the last decades. (c)+(d): Absolute (c) and relative (d) SOC stocks compared to a potential natural state showing different hot spots of SOC dynamics. Whereas the absolute losses might be in temperate dry regions, relative losses are more prominent in tropical moist areas. (BB: Increase map size, legend is larger than map, no need for that)}\label{fig:SOCmaps}
\end{figure}

In fig.~1(a) we provide the first world map of SOC on croplands
considering real world management data on the global scale. Our
spatially explicit results moreover show hot spots of SOC losses as well
as gains in two different ways: 1. Absolute SOC changes (see fig.~1(c))
indicate areas with high importance for the global SOC emissions. The
might be driven by huge relative losses or a high natural stock, from
which even small deviations could lead to substantial losses. 2. To
attribute SOC losses to insufficient agricultural management relative
SOC changes (\(F^{SCF}\), see fig.~1(d)) are a helpful tool. They
indicate areas with huge difference in carbon inflows or SOC decay
compared to natural vegetation, that might be overcome due to improved
agricultural pratices.

\subsection{Agricultural management effects on SOC emissions and
cycling}

Global cummulative SOC emissions are decreasing (see fig.~1(c)). Fig. 2
reveals the relative impact of management effects by freezing tillage
areas as well as carbon inflows from residues or manure at the level of
1975. Our counterfactual scenarios show that the increasing residue
carbon input had the biggest overall effect on SOC stocks. Without
changes in management regimes especially in residue inflows to the soil,
global cummulative SOC emissions would still grow. The strong effect of
carbon residue amounts are also visible in the carbon flow diagram for
agricultural production for the year 2010 (see fig.~3)

\begin{figure}[H]
\includegraphics[width=18cm]{../ResultNotebooks/Output/Images/scenario} \caption{Global SOC gap in GtC for various stylized management counterfactual scenarios compared to the modeled historical baseline. (BB: order of legend should be equal to order of lines. Management=tillage+reside+manure?)}\label{fig:SOCscen}
\end{figure}

\begin{figure}[H]
\includegraphics[width=16cm]{../ResultNotebooks/Output/Images/OuFlowFig} \caption{Global carbon flows (small numbers) and stocks (bold numbers) within the agricultural system for the year 2010 (in MtC): Most important carbon sources on cropland are crop residues. Note the two numbers on carbon inputs to soil denote carbon applied to the field and carbon entering the soil (difference is decomposted before official counting as soil).(BB: adjuste flow size to numbers, include arrows for losses, e.g. --> x ,otherwise its confusing, trees need black outer line, roots have no clean upper end)}\label{fig:FlowFig}
\end{figure}

\newpage

\newpage

\section{Discussion}

\subsection{Including agricultural management data changes the sign of
the trend}

This study provides an analysis of historic SOC stock changes on
cropland. We determine the SOC trends on cropland compared to a counterfactual scenario with a world under natural vegetation under identical historical climatic
conditions (\(SOC_{natveg}\)). Our results show, that human activity lead to cummulative SOC emissions
of around 37 GtC in 2010. Whereas recent modelling
estimates of global SOC emissions indicate an ongoing increase of SOC
emissions (Sanderman et al, Pugh et al.), our study indicates that the
global SOC gap is slowly closing due to improved management.

According to Sanderman et al.~cummulative SOC emissions since beginning
of human cropping activities have been at around 37 GtC for the first 30
cm of the soil with half of it attributed to grazing. It was also
pointed out that these results might be conservatively low
compared to experimental results. Considering the huge uncertainties in modelling SOC on the global scale, 
our estimate of 37 GtC in 2010 for cropland emissions only still seems consistent with Sanderman et al.~estimates.

Furthermore (BB: consider using a spelling checker) the results of Sanderman et al.~calculated (BB: the results calculated?) historical trends
based on agricultural land expansion without considering SOC variations
due to different management systems at all. Pugh et al.~ considered management
effects like tillage and residue recycling in a static way, but
neither changes over time nor alignment to oberserved historical data
like yields-levels or no-tillage areas were taken into account. Under
these assumptions this study only found marginal effects (BB: be more precise) of crop
productivity and other management effects on cummulative SOC emissions.

Our study for the first time uses a dynanmic management dataset as driver for SOC dynamics. We
show that the moderate global cropland expansion of around 11\% between 1974 and
2010 and the resulting depletion of SOC stocks in converted cropland has been outweighted by improved agricultural yields ad practises.
Moreover our sensitivity analysis indicades that (1) yield increases and
with that the increase in residue biomass might play the most important
role, followed by (2) enhanced residue recycling rates, (3) improved
manure recycling (e.g.~due to improved animal waste management system)
and (4) the adoption of no tillage practises.

Modelling management effects on the global scale comes however with 
parametric and structural uncertainties. As pointed out by Keel
(2017) and Smith (2019), carbon input
calculations are highly sensitive to the choice of allometric functions
determining below and above ground residue estimates from harvested
quantities. Keel et al.~questioned whether below ground residues might
increase with a fixed root:shoot ratio rather than being independend from
productivity gains. Following this argumentation SOC results shown in
this study might especially in high-yielding farming system overestimate
actual SOC stocks (BB: No. Very likely, root:shoot ratios come from developed countries. They rather underestimate SOC in LICs). However (?), according to our study above ground residue biomass recycling seems to contribute even more to overall SOC stocks due
to higher input rates.

\subsection{Modeled management effect in line with default IPCC
assumptions}

To validate the effect of our modelled SOC stocks and stock changes
under management, we compare our results to default IPCC stock changes
factors (cite) which are based on measurement data for croplands (see
@ref(tab:SCFtable)).

\begin{figure}[H]
\includegraphics[width=16cm]{../ResultNotebooks/Output/Images/TableSCF_comparison} \caption{This table shows different estimates of stock change factors $F_{SCF}$, defined as the soil carbon stock of managed croplands relative to the counterfactual soil carbon stock of undisturbed natural vegetation. We compare average values for the four IPCC climate zone classifications ... (BB: Your caption has to be self-explaining for people who do not know the rest of your article. All elements have to be defined and explained.)IPCC 2006 as well as 2019 default (medium input) and low input regimes(BB:regimes?) stock change factors  without other subsystem consideration compared results of this study (SOC budget) for 1990 and 2010.}\label{fig:SCFtable}
\end{figure}

We also intercompared our stock-change factors \(F_{SCF}\), which describe the ratio between ... . 
To allow for comparison, we aggregated our stock change factors to the four IPCC climate zones (see table X).
Our estimates correspond very well to the default stock-change factors used for the tier 1 estimation of IPCC, 2006. For the tropical
regions the assumptions changed notably from the guidelines in 2006 to
the update in 2019 (BB: do you have any indication why they changed so dramatically?), leaving our results too low in comparison with IPCC,
2019. The development of \(F_{SCF}\) over time in our study shows the substantial impact of changed management.
Considering yield gaps in mainly developing regions in the tropics
the default assumption of medium input systems, might be an
overestimation of actual SOC state. The additional effect of considering
low inpout regimes in tropical regions can however not explain the full
mismatch to IPCC 2019 values but account for at least 5-7\% of it.

\subsection{SOC stocks inline with literature}

The worlds SOC stock and its changes are highly uncertain (cite), which
is visible by the wide range of global SOC stock estimates (see
@ref(tab:SOCtable)).

\begin{figure}[H]
\includegraphics[width=8cm]{../ResultNotebooks/Output/Images/TableSOC_comparison} \caption{Modelled as well as data based estimation for global SOC stock (for 2010) in GtC for the first 30 cm of soil.(BB: total land or just croplands? All other studies are for 2010?)}\label{fig:SOCtable}
\end{figure}

(BB: do these studies assume landuse? what do they do for urban land?)

The global esimates of SOC stock by this study are on the lower end
compared to other modelled results or more data driven estimates.
Looking on regional results (fig.~SX in supplement), our estimates turn
out to be in good agreement for most regions with the largest deviations
for boreal areas. Considering that the model was parametrized for
croplands, these mismatches are not superising since the temperature
effects on decomposition are fundamentally different for
permaforst soils. To avoid that this bias
influences our results, our study focusses exclusively on cropland
soils, excluding most of the boreal zone. Morover, when focussing on SOC
changes, pristine natural vegetated areas without human land management
under the same historic climatic conditions cancel out in the
calculation of SOC emissions.

Our estimates for total SOC stocks of the world, as well as our SOC initialization are dominated by the representation of natural lands (BB: and pastures?), which are however only estimated in a basic manner. For example, we do not have a differentiated parametrization of nitrogen and lignin content of litterfall.
This leaves carbon inputs and decay behaviour for natural land rather
uncertain. (delete: As all results are valued against these potential natural
SOC - BB: the valuation against the natural state is actually making the results more robust. We are only looking at the differences caused by management, not on the absolute states. Make this clear here again (even though you already mention it in the previous paragraph in a differnet context)
The absolute values of stocks and emissions from land-use change therefore have to be used
with caution. Especially in less forested areas the natural land
representation might be off, due to parameterization assumptions of the
natural litterfall. Nevertheless total SOC stocks are in a reasonable
range and the stock change factors \(F^{SCF}\) are in good agreement
with the Tier 1 default values of 2006. Moreover, emissions or carbon sinks
from changed management are not affected from the natural vegetation representation.
To identify the importance of lignin and nitrogen parameterization of natural
litterfall, we conducted a sensitivity analysis (see appendix). It shows that the general trend of decreasing
SOC emissions is not altered even by rather unlikely parameter choices (BB: Why do you look at unlikely parameter choices???).

\subsection{Important short commings --\textgreater{} move to appendix}

(BB: Or include into outlook)
Smaller points and shortcommings:

-This study as the IPCC guidelines suggested has limited her focus to
the first 30 cm of the soil profile, leaving changes in the subsoil
unnoticed. Nevertheless studies (see Don on tillage) have shown, the
subsoil to be a game changer in evaluating total SOC losses or gains for
no-tillage systems. It has been argued that for intensivly tilled soils,
subsoil SOC is increasing due to the import of carbon rich topsoil to
deeper soil layers. Following these argumentation SOC stocks in
croplands might even be underestimated.

\begin{itemize}
\item
  Fertilizer interaction is not included here by accounting for
  additional N supply that would alter C:N ratio of the carbon inputs.
  Tier 2 steady-state method is neglecting fertilizer application,
  however we would have fertilizer ammounts at hand to include them, if
  proper representation of fertilizer within the method would be
  possible to add.
\item
  Pasture dynamics are neglected and treated as natural vegetation,
  which might be -- looking on pasture degradation due to overgrazing --
  oversimplified for some spots, but is inline with assumption on
  pasture SOC stocks done before (see Tier 1 IPCC). (Note that also
  manure excreted to pastures is neglected within these analysis, since
  we focus purely on cropland dynamics.)
\item
  No tillage adaptaion is neglected on cropland due to less common
  adoption of no tillage and conservation agriculture. Pastures are
  assumed to be not tilled at all (propably only heavy managed pastures
  are tilled with some rotation)
\item
  Irrigated areas are not crop specific and irrigation is not restricted
  to growing periods (since it is very complex to calculate average
  growing periods). Crop specific growing periods might be possible
  using LPJmL data.
\item
  flooded rice area are not represented correctly as parameterization
  does not hold true for flooded conditions.
\item
  Carbon displacement via leaching and erosion is neglected in this
  study.
\item
  Non-net/Gross land use transitions are not tracked in this study.
\item
  Within cropland we do not track croparea transitions, but rather look
  at statistical distributions of the crop functional types. Due to crop
  rotations and missing data on crop specific distributions, these
  transitions would be any way rather uncertain.
\item
  The disaggregation of manure to uild-up areas (in the case of
  extensive monograstrics) is leeding to a lot of displaced manure (?)
  that is cut off
\item
  It is known that there are mismatches between FAO statistics and LUH
  areas. As far as possibles there were harmonized within this study.
\item
  ``The Tier 2 method does not simulate C change but simply calculates
  an annual C stock change from the current C stock to the future
  steady-state soil C stock calculated based on current conditions.''
  Leading to the fact that our total stock results are highly
  uncertained. \newpage
\end{itemize}

\conclusions

Outlook including perspective on mitigation (and soc enhancement in the
future). \newpage

\section{Appendix}

\hypertarget{append:subsection2mrfunctions}{%
\subsection{table on method subsections to functions within R
packages}\label{append:subsection2mrfunctions}}

\hypertarget{append:Tableluh2fao2mag}{%
\subsection{table on mapping
LUH2FAO2MAG}\label{append:Tableluh2fao2mag}}

\hypertarget{append:Tablekcr2kres}{%
\subsection{kcr2kres mapping}\label{append:Tablekcr2kres}}

\hypertarget{append:Tablec2dm}{%
\subsection{carbon 2 dry matter}\label{append:Tablec2dm}}

Litter is coming from LPJmL in carbon units - transformation with 0.44
is done twice reverting the effect of the transformation

\hypertarget{append:TableclossAWMS}{%
\subsection{closs in AWMS - Table}\label{append:TableclossAWMS}}

\hypertarget{append:climatemap}{%
\subsection{map on climate zone used for SCF}\label{append:climatemap}}

\newpage




\codedataavailability{Software code for paper and result prepartion can
be found under www.github.com/k4rst3ns/. Data used for the output can be
found under
.} %% use this section when having data sets and software code available



%%%%%%%%%%%%%%%%%%%%%%%%%%%%%%%%%%%%%%%%%%
%% optional

%%%%%%%%%%%%%%%%%%%%%%%%%%%%%%%%%%%%%%%%%%

%%%%%%%%%%%%%%%%%%%%%%%%%%%%%%%%%%%%%%%%%%
\authorcontribution{Karstens wrote code and paper build on work of
Bodirsky (and ). Bodirsky, and Popp revised paper.} %% optional section

%%%%%%%%%%%%%%%%%%%%%%%%%%%%%%%%%%%%%%%%%%
\competinginterests{The authors declare no competing
interests.} %% this section is mandatory even if you declare that no competing interests are present

%%%%%%%%%%%%%%%%%%%%%%%%%%%%%%%%%%%%%%%%%%
\disclaimer{We like Copernicus.} %% optional section

%%%%%%%%%%%%%%%%%%%%%%%%%%%%%%%%%%%%%%%%%%
\begin{acknowledgements}
Thanks to the rticles contributors!
\end{acknowledgements}

%% REFERENCES
%% DN: pre-configured to BibTeX for rticles

%% The reference list is compiled as follows:
%%
%% \begin{thebibliography}{}
%%
%% \bibitem[AUTHOR(YEAR)]{LABEL1}
%% REFERENCE 1
%%
%% \bibitem[AUTHOR(YEAR)]{LABEL2}
%% REFERENCE 2
%%
%% \end{thebibliography}

%% Since the Copernicus LaTeX package includes the BibTeX style file copernicus.bst,
%% authors experienced with BibTeX only have to include the following two lines:
%%
\bibliographystyle{copernicus}
\bibliography{SOCbudget.bib}
%%
%% URLs and DOIs can be entered in your BibTeX file as:
%%
%% URL = {http://www.xyz.org/~jones/idx_g.htm}
%% DOI = {10.5194/xyz}


%% LITERATURE CITATIONS
%%
%% command                        & example result
%% \citet{jones90}|               & Jones et al. (1990)
%% \citep{jones90}|               & (Jones et al., 1990)
%% \citep{jones90,jones93}|       & (Jones et al., 1990, 1993)
%% \citep[p.~32]{jones90}|        & (Jones et al., 1990, p.~32)
%% \citep[e.g.,][]{jones90}|      & (e.g., Jones et al., 1990)
%% \citep[e.g.,][p.~32]{jones90}| & (e.g., Jones et al., 1990, p.~32)
%% \citeauthor{jones90}|          & Jones et al.
%% \citeyear{jones90}|            & 1990

\end{document}
