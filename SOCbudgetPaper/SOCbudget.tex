%% Copernicus Publications Manuscript Preparation Template for LaTeX Submissions
%% ---------------------------------
%% This template should be used for copernicus.cls
%% The class file and some style files are bundled in the Copernicus Latex Package, which can be downloaded from the different journal webpages.
%% For further assistance please contact Copernicus Publications at: production@copernicus.org
%% https://publications.copernicus.org/for_authors/manuscript_preparation.html

%% copernicus_rticles_template (flag for rticles template detection - do not remove!)

%% Please use the following documentclass and journal abbreviations for discussion papers and final revised papers.

%% 2-column papers and discussion papers
\documentclass[gc, manuscript]{copernicus}



%% Journal abbreviations (please use the same for discussion papers and final revised papers)


% Advances in Geosciences (adgeo)
% Advances in Radio Science (ars)
% Advances in Science and Research (asr)
% Advances in Statistical Climatology, Meteorology and Oceanography (ascmo)
% Annales Geophysicae (angeo)
% Archives Animal Breeding (aab)
% ASTRA Proceedings (ap)
% Atmospheric Chemistry and Physics (acp)
% Atmospheric Measurement Techniques (amt)
% Biogeosciences (bg)
% Climate of the Past (cp)
% DEUQUA Special Publications (deuquasp)
% Drinking Water Engineering and Science (dwes)
% Earth Surface Dynamics (esurf)
% Earth System Dynamics (esd)
% Earth System Science Data (essd)
% E&G Quaternary Science Journal (egqsj)
% Fossil Record (fr)
% Geochronology (gchron)
% Geographica Helvetica (gh)
% Geoscience Communication (gc)
% Geoscientific Instrumentation, Methods and Data Systems (gi)
% Geoscientific Model Development (gmd)
% History of Geo- and Space Sciences (hgss)
% Hydrology and Earth System Sciences (hess)
% Journal of Micropalaeontology (jm)
% Journal of Sensors and Sensor Systems (jsss)
% Mechanical Sciences (ms)
% Natural Hazards and Earth System Sciences (nhess)
% Nonlinear Processes in Geophysics (npg)
% Ocean Science (os)
% Primate Biology (pb)
% Proceedings of the International Association of Hydrological Sciences (piahs)
% Scientific Drilling (sd)
% SOIL (soil)
% Solid Earth (se)
% The Cryosphere (tc)
% Web Ecology (we)
% Wind Energy Science (wes)


%% \usepackage commands included in the copernicus.cls:
%\usepackage[german, english]{babel}
%\usepackage{tabularx}
%\usepackage{cancel}
%\usepackage{multirow}
%\usepackage{supertabular}
%\usepackage{algorithmic}
%\usepackage{algorithm}
%\usepackage{amsthm}
%\usepackage{float}
%\usepackage{subfig}
%\usepackage{rotating}


% The "Technical instructions for LaTex" by Copernicus require _not_ to insert any additional packages.
%
\usepackage{algorithmic}
\usepackage{algorithm}


\begin{document}

\title{Historical Soil Organic Carbon Budget}


\Author[1]{Kristine}{Karstens}
\Author[1]{Benjamin Leon}{Bodirsky}
\Author[1]{Alexander}{Popp}


\affil[1]{Potsdam-Institut of Climate Impacts Research, Potsdam, Germany}

%% The [] brackets identify the author with the corresponding affiliation. 1, 2, 3, etc. should be inserted.



\runningtitle{R Markdown Template for Copernicus}

\runningauthor{Nüst et al.}


\correspondence{Kristine\ Karstens\ (kristine.karstenst@pik-potsdam.de)}



\received{}
\pubdiscuss{} %% only important for two-stage journals
\revised{}
\accepted{}
\published{}

%% These dates will be inserted by Copernicus Publications during the typesetting process.


\firstpage{1}

\maketitle


\begin{abstract}
SOC one of larges c sinks on earth (3 times larger biospehre pool).
Agricultural management leads to a depletion of soil organic crabon.
However this depletion of soil organic carbon (SOC) pools are so far not
well represented in global assessments of historic carbon emissions.
While SOC models often represent well the biochemical processes that
lead to the accumulation and decay of SOC, the management decisions
driving these biophysical processes are still little investigated. Here
we create a spatial explicit data set for crop residue and manure
management on cropland based on global historic production (FAOSTAT) and
land-use (LUH2) data and combine it with the IPCC Tier 2 approach to
create a half-degree resolution soil organic carbon budget on mineral
soils. We estimate that due to arable farming soils have lost over (?)
GtOC of which (??) GtOC have been released within the period 1990-2010.
Tier 2 IPCC methodolgy estimates higher soil organic carbon losses than
Tier 1 methods, which may origin from \ldots{} . We also find that SOC
is very sensity to management decision such as residue recycling
indicating the nessessity to incorporated better management data in soil
models.
\end{abstract}


\copyrightstatement{The author's copyright for this publication is transferred to
institution/company.}


\introduction

Introduction text goes here. You can change the name of the section if
neccessary using
\texttt{\textbackslash{}introduction{[}modified\ heading{]}}.

\section{Method (50)}

\subsection{Carbon Budget (50)}

Based on the IPCC guidelines vol.~4 and their refinement
(\citep{ipcc_2006_2006},\citep{ipcc_2019_2019}) for soil organic carbon
stocks, we combining approaches to estimate SOC stocks by weighting
inflows via dead plant material (see \ref{sec:carboninputs}) against
outflows through SOC decay (see \ref{sec:tier2}). Carbon displacement
via leaching and erosion is neglected in this study. We calculate annual
land use type specific budgets for cropland, pastures and natural
vegetation, also representin land conversion as transfer between landuse
type budgets. A simple approach based o the tier 1 method of the older
IPCC guidelines vol 4. (\citet{ipcc_2006_2006}) using stock change
factors, is applied to cross validated results \ref{sec:tier1}

\subsubsection{Carbon Inputs to the Soil}\label{sec:carboninputs}

Carbon input estimations are based on the land use type. Whereas
cropland inputs are mainly formed by disaggregated country statistics on
residue, dead below ground and cover crop biomass, pasture and natural
vegetation inputs are estimate via modelled annual litterfall rates.
Using the steady-state method of the IPCC guidelines
(\citep{ipcc_2019_2019}) carbon inputs have to be accompined by data on
lignin and nitrogen content to allocate dead plant biomass to the
corrosponding soil pools based on the chemical texture. Sources for all
use data can be found in table \ref{tab:datasourceinputs}

\begin{table*}[h]
\caption{Sources for carbon input data}
\begin{tabular}{l l l}
\tophline
 Land use types   & carbon inputs & nitrogen and lignin content \\
\middlehline
 Cropland         & FAO statistics, AQUASTAT, LPJmL4 [1] & default values [\cite{ipcc_2006_2006}] \\
 Pasture          & annual litterfall in $\tfrac{gC}{m^2}$ from LPJmL4 - manage grassland [3] & default values [2] \\
 Natural vegetation & annual litterfall in $\tfrac{gC}{m^2}$ from LPJmL4  & \begin{minipage}[t]{0.37\columnwidth}\raggedright\strut Nitrogen and lignin content of tree compartments used in CENTURY \strut \end{minipage}\tabularnewline
\bottomhline
\end{tabular}
\label{tab:datasourceinputs}
\belowtable{}
\end{table*}

\subsubsection{Soil Carbon turnover following (new) Tier 2 method
(300)}\label{sec:tier2}

We are following the steady-state method of the refinement of the IPCC
guidelines vol.~4 (\citet{ipcc_2019_2019}) by calculating yearly
turnover and transfer rates between three different SOC pools for the
topsoil (0-30 cm). The approach is based on global parameters
({[}@ref(ipcc\_2019\_2019){]}) as well as half-degree data on sand
fraction (SoilGrids), temperature, preciptation and potential
evapotranspiration (CRU). Next to the given climatic and natural
biophysical properties irrigation regime (rainfed vs.~irrigated) as well
as tillage (as soil disturbance indicator) modificate processes. For
cropland an assessment of tillage types and irrigation conditions has
been made, whereas on pastures and natural vegetation, we assume rainfed
and non-tilled conditions.

\subsubsection{Soil Carbon turnover following Tier 1
(150)}\label{sec:tier1}

Following the tier 1 approach of the IPCC guidelines vol.~4
(\citet{ipcc_2006_2006}), stocks are estimated via stock change factors
given by the IPCC for the topsoil (0-30 cm) and based on measurements.
The factors are differentiate between different crop and management
systems reflecting different dynamics under changed in- and outflows
without explicitly tracking these. They can be seen as conservative
guesses and will be used to evaluate our modelling based results.

\subsection{Agricultural management (50)}\label{sec:agrimanagement}

We combine data sets to estimate agricultural flows and management
decisions on cropland.

\subsubsection{Landuse and Landuse Change (150)}

We use LUH2v2 data for major Landuse types and their transition and fit
cropspecific areas to country scale FAO data.

\subsubsection{Crop Production and Residues (300)}

FAO Production values are combined with Feed estimations from
{[}Isabelles Paper{]} and rule based demand shares. LPJmL yield and LUH
landuse patterns are used to scale down to half-degree.

\subsubsection{Livestock Distribution and Manure Excretion (300)}

Based on {[}Gridded Livestock of the world{]} we use rule based
asumption to estimate livestock and manure distribution on the globe.
Animal waste system shares are used as is {[}Bodirsky{]}.

\subsubsection{Irrigation (100)}

Simple growing period calculations together with irrigation shares of
LUH2v2 are use to estimate water effects on decay rates.

\subsubsection{Tillage (100)}

Tillage data sets of {[}Vera, others{]} together with rules are used to
drive tillage effect on decay rates.

\section{Results}

\section{Discussion}

\conclusions

The conclusion goes here. You can modify the section name with
\texttt{\textbackslash{}conclusions{[}modified\ heading\ if\ necessary{]}}.




\codedataavailability{use this to add a statement when having data sets and software code
available} %% use this section when having data sets and software code available



%%%%%%%%%%%%%%%%%%%%%%%%%%%%%%%%%%%%%%%%%%
%% optional

%%%%%%%%%%%%%%%%%%%%%%%%%%%%%%%%%%%%%%%%%%
\appendix
\section{Figures and tables in appendices}\subsection{Option 1}

If you sorted all figures and tables into the sections of the text,
please also sort the appendix figures and appendix tables into the
respective appendix sections. They will be correctly named
automatically. \subsection{Option 2} If you put all figures after the
reference list, please insert appendix tables and figures after the
normal tables and figures.

\texttt{\textbackslash{}appendixfigures} needs to be added in front of
appendix figures \texttt{\textbackslash{}appendixtables} needs to be
added in front of appendix tables

Please add \texttt{\textbackslash{}clearpage} between each table and/or
figure. Further guidelines on figures and tables can be found below.
Regarding figures and tables in appendices, the following two options
are possible depending on your general handling of figures and tables in
the manuscript environment: To rename them correctly to A1, A2, etc.,
please add the following commands in front of them:
\noappendix

%%%%%%%%%%%%%%%%%%%%%%%%%%%%%%%%%%%%%%%%%%
\authorcontribution{Karstens wrote code and paper build on work of Bodirsky. Bodirsky and
Popp revised paper.} %% optional section

%%%%%%%%%%%%%%%%%%%%%%%%%%%%%%%%%%%%%%%%%%
\competinginterests{The authors declare no competing interests.} %% this section is mandatory even if you declare that no competing interests are present

%%%%%%%%%%%%%%%%%%%%%%%%%%%%%%%%%%%%%%%%%%
\disclaimer{We like Copernicus.} %% optional section

%%%%%%%%%%%%%%%%%%%%%%%%%%%%%%%%%%%%%%%%%%
\begin{acknowledgements}
Thanks to the rticles contributors!
\end{acknowledgements}

%% REFERENCES
%% DN: pre-configured to BibTeX for rticles

%% The reference list is compiled as follows:
%%
%% \begin{thebibliography}{}
%%
%% \bibitem[AUTHOR(YEAR)]{LABEL1}
%% REFERENCE 1
%%
%% \bibitem[AUTHOR(YEAR)]{LABEL2}
%% REFERENCE 2
%%
%% \end{thebibliography}

%% Since the Copernicus LaTeX package includes the BibTeX style file copernicus.bst,
%% authors experienced with BibTeX only have to include the following two lines:
%%
\bibliographystyle{copernicus}
\bibliography{SOCbudget.bib}
%%
%% URLs and DOIs can be entered in your BibTeX file as:
%%
%% URL = {http://www.xyz.org/~jones/idx_g.htm}
%% DOI = {10.5194/xyz}


%% LITERATURE CITATIONS
%%
%% command                        & example result
%% \citet{jones90}|               & Jones et al. (1990)
%% \citep{jones90}|               & (Jones et al., 1990)
%% \citep{jones90,jones93}|       & (Jones et al., 1990, 1993)
%% \citep[p.~32]{jones90}|        & (Jones et al., 1990, p.~32)
%% \citep[e.g.,][]{jones90}|      & (e.g., Jones et al., 1990)
%% \citep[e.g.,][p.~32]{jones90}| & (e.g., Jones et al., 1990, p.~32)
%% \citeauthor{jones90}|          & Jones et al.
%% \citeyear{jones90}|            & 1990

\end{document}
