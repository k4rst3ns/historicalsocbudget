%% Copernicus Publications Manuscript Preparation Template for LaTeX Submissions
%% ---------------------------------
%% This template should be used for copernicus.cls
%% The class file and some style files are bundled in the Copernicus Latex Package, which can be downloaded from the different journal webpages.
%% For further assistance please contact Copernicus Publications at: production@copernicus.org
%% https://publications.copernicus.org/for_authors/manuscript_preparation.html

%% copernicus_rticles_template (flag for rticles template detection - do not remove!)

%% Please use the following documentclass and journal abbreviations for discussion papers and final revised papers.

%% 2-column papers and discussion papers
\documentclass[gc, manuscript]{copernicus}



%% Journal abbreviations (please use the same for discussion papers and final revised papers)


% Advances in Geosciences (adgeo)
% Advances in Radio Science (ars)
% Advances in Science and Research (asr)
% Advances in Statistical Climatology, Meteorology and Oceanography (ascmo)
% Annales Geophysicae (angeo)
% Archives Animal Breeding (aab)
% ASTRA Proceedings (ap)
% Atmospheric Chemistry and Physics (acp)
% Atmospheric Measurement Techniques (amt)
% Biogeosciences (bg)
% Climate of the Past (cp)
% DEUQUA Special Publications (deuquasp)
% Drinking Water Engineering and Science (dwes)
% Earth Surface Dynamics (esurf)
% Earth System Dynamics (esd)
% Earth System Science Data (essd)
% E&G Quaternary Science Journal (egqsj)
% Fossil Record (fr)
% Geochronology (gchron)
% Geographica Helvetica (gh)
% Geoscience Communication (gc)
% Geoscientific Instrumentation, Methods and Data Systems (gi)
% Geoscientific Model Development (gmd)
% History of Geo- and Space Sciences (hgss)
% Hydrology and Earth System Sciences (hess)
% Journal of Micropalaeontology (jm)
% Journal of Sensors and Sensor Systems (jsss)
% Mechanical Sciences (ms)
% Natural Hazards and Earth System Sciences (nhess)
% Nonlinear Processes in Geophysics (npg)
% Ocean Science (os)
% Primate Biology (pb)
% Proceedings of the International Association of Hydrological Sciences (piahs)
% Scientific Drilling (sd)
% SOIL (soil)
% Solid Earth (se)
% The Cryosphere (tc)
% Web Ecology (we)
% Wind Energy Science (wes)


%% \usepackage commands included in the copernicus.cls:
%\usepackage[german, english]{babel}
%\usepackage{tabularx}
%\usepackage{cancel}
%\usepackage{multirow}
%\usepackage{supertabular}
%\usepackage{algorithmic}
%\usepackage{algorithm}
%\usepackage{amsthm}
%\usepackage{float}
%\usepackage{subfig}
%\usepackage{rotating}


% The "Technical instructions for LaTex" by Copernicus require _not_ to insert any additional packages.
%
\usepackage{algorithmic}
\usepackage{algorithm}

\usepackage{color}
\usepackage{fancyvrb}
\newcommand{\VerbBar}{|}
\newcommand{\VERB}{\Verb[commandchars=\\\{\}]}
\DefineVerbatimEnvironment{Highlighting}{Verbatim}{commandchars=\\\{\}}
% Add ',fontsize=\small' for more characters per line
\usepackage{framed}
\definecolor{shadecolor}{RGB}{248,248,248}
\newenvironment{Shaded}{\begin{snugshade}}{\end{snugshade}}
\newcommand{\AlertTok}[1]{\textcolor[rgb]{0.94,0.16,0.16}{#1}}
\newcommand{\AnnotationTok}[1]{\textcolor[rgb]{0.56,0.35,0.01}{\textbf{\textit{#1}}}}
\newcommand{\AttributeTok}[1]{\textcolor[rgb]{0.77,0.63,0.00}{#1}}
\newcommand{\BaseNTok}[1]{\textcolor[rgb]{0.00,0.00,0.81}{#1}}
\newcommand{\BuiltInTok}[1]{#1}
\newcommand{\CharTok}[1]{\textcolor[rgb]{0.31,0.60,0.02}{#1}}
\newcommand{\CommentTok}[1]{\textcolor[rgb]{0.56,0.35,0.01}{\textit{#1}}}
\newcommand{\CommentVarTok}[1]{\textcolor[rgb]{0.56,0.35,0.01}{\textbf{\textit{#1}}}}
\newcommand{\ConstantTok}[1]{\textcolor[rgb]{0.00,0.00,0.00}{#1}}
\newcommand{\ControlFlowTok}[1]{\textcolor[rgb]{0.13,0.29,0.53}{\textbf{#1}}}
\newcommand{\DataTypeTok}[1]{\textcolor[rgb]{0.13,0.29,0.53}{#1}}
\newcommand{\DecValTok}[1]{\textcolor[rgb]{0.00,0.00,0.81}{#1}}
\newcommand{\DocumentationTok}[1]{\textcolor[rgb]{0.56,0.35,0.01}{\textbf{\textit{#1}}}}
\newcommand{\ErrorTok}[1]{\textcolor[rgb]{0.64,0.00,0.00}{\textbf{#1}}}
\newcommand{\ExtensionTok}[1]{#1}
\newcommand{\FloatTok}[1]{\textcolor[rgb]{0.00,0.00,0.81}{#1}}
\newcommand{\FunctionTok}[1]{\textcolor[rgb]{0.00,0.00,0.00}{#1}}
\newcommand{\ImportTok}[1]{#1}
\newcommand{\InformationTok}[1]{\textcolor[rgb]{0.56,0.35,0.01}{\textbf{\textit{#1}}}}
\newcommand{\KeywordTok}[1]{\textcolor[rgb]{0.13,0.29,0.53}{\textbf{#1}}}
\newcommand{\NormalTok}[1]{#1}
\newcommand{\OperatorTok}[1]{\textcolor[rgb]{0.81,0.36,0.00}{\textbf{#1}}}
\newcommand{\OtherTok}[1]{\textcolor[rgb]{0.56,0.35,0.01}{#1}}
\newcommand{\PreprocessorTok}[1]{\textcolor[rgb]{0.56,0.35,0.01}{\textit{#1}}}
\newcommand{\RegionMarkerTok}[1]{#1}
\newcommand{\SpecialCharTok}[1]{\textcolor[rgb]{0.00,0.00,0.00}{#1}}
\newcommand{\SpecialStringTok}[1]{\textcolor[rgb]{0.31,0.60,0.02}{#1}}
\newcommand{\StringTok}[1]{\textcolor[rgb]{0.31,0.60,0.02}{#1}}
\newcommand{\VariableTok}[1]{\textcolor[rgb]{0.00,0.00,0.00}{#1}}
\newcommand{\VerbatimStringTok}[1]{\textcolor[rgb]{0.31,0.60,0.02}{#1}}
\newcommand{\WarningTok}[1]{\textcolor[rgb]{0.56,0.35,0.01}{\textbf{\textit{#1}}}}

\begin{document}

\title{Historical Soil Organic Carbon Budget}


\Author[1]{Kristine}{Karstens}
\Author[1]{Benjamin Leon}{Bodirsky}
\Author[1]{Alexander}{Popp}


\affil[1]{Potsdam-Institut of Climate Impacts Research, Potsdam,
Germany}

%% The [] brackets identify the author with the corresponding affiliation. 1, 2, 3, etc. should be inserted.



\runningtitle{R Markdown Template for Copernicus}

\runningauthor{Nüst et al.}


\correspondence{Kristine\ Karstens\ (kristine.karstenst@pik-potsdam.de)}



\received{}
\pubdiscuss{} %% only important for two-stage journals
\revised{}
\accepted{}
\published{}

%% These dates will be inserted by Copernicus Publications during the typesetting process.


\firstpage{1}

\maketitle


\begin{abstract}
SOC one of larges c sinks on earth (3 times larger biosphere pool).
Agricultural management leads to a depletion of soil organic carbon.
However this depletion of soil organic carbon (SOC) pools are so far not
well represented in global assessments of historic carbon emissions.
While SOC models often represent well the biochemical processes that
lead to the accumulation and decay of SOC, the management decisions
driving these biophysical processes are still little investigated. Here
we create a spatial explicit data set for crop residue and manure
management on cropland based on global historic production (FAOSTAT) and
land-use (LUH2) data and combine it with the IPCC Tier 2 approach to
create a half degree resolution soil organic carbon budget on mineral
soils. We estimate that due to arable farming soils have lost over (?)
GtOC of which (??) GtOC have been released within the period 1990-2010.
We show that, our results on global scale based on Tier 2 IPCC
methodolgy are in good agreement with Tier 1 default assumptions. We
also find that SOC is very sensitive to management decision such as
residue recycling indicating the nessessity to incorporated better
management data in soil models.
\end{abstract}


\copyrightstatement{The author's copyright for this publication is
transferred to institution/company.}


\newpage

\introduction

Within the last centuries the recognition of soil organic carbon (SOC)
has developed from being a major source of agricultural carbon emissions
towards a potentially large additional sink to offset other green house
gas emissions and thus mitigate climate change. Whereas the extent of
sink potential by SOC enhancement is highly debated (add refs), it is
largely agreed upon that the SOC pool itself is the biggest terrestrial
carbon pool, over topping the atmospheric and even more the biospheric
carbon pool multiple times. Even small changes in SOC drivers might lead
to substantial shifts in earth carbon cycled and influence the
atmospheric CO2 concentration (ref. permafrost melting). The specific
amount of carbon stored in the soil is uncertain though and estimates
ranging from 1500 to 2400 GtC for the first meter of the soil profile
(Bathjes, 1996).

Mapping the worlds SOC has been of great interest over the last decades
and led to an increased quality of SOC maps as well as a even better
understanding of factors driving magnitude, distribution and dynamics of
SOC pools. It is undisputed that natural properties like climatic,
biophysical and landscape characteristics play the most important role
in this regard. These factors can not be easily altered by human
intervention, leaving land cover, land use and land management changes
the most important factors for SOC dynamics till dawn of civilization .

Recent studies identified the SOC debt of anthropogenic source at around
116 GtC (Sanderman et al.), which compares to older estimates of around
60-130 GtC (Lal, 2006). Other studies have focus more closely on
spatially disaggregation SOC changes via advanced digital soil mapping
techniques (S-World; Stoorvogel 2, 2017) or better representation of
biogeochemcial processes within SOC dynamics (). Most of these studies
doing cutting edge research in their field, but seem to lack proper
representation of available information on land management.

On the other hand small-scale models (ref. Daycent, RothC, Ecosse,
C-Tool) are able to represent fairly well SOC dynamics on field-scale
driven by high quality management data on yield levels, fertilizer
inputs and various other on farming activities. Due to the lack of
proper, concise management data to feed these input demanding models on
larger extends, it seems still very complex to scale them up to a global
extent.

This study however wants to contribute filling the gap on impacts of
agricultural management like yield levels, residue recycling, manure
amendments, irrigation and tillage on SOC stocks of the worlds
croplands. It is doing so by combining work on agricultural management
data on global level with a robust but light weighted SOC model to
estimate SOC stocks and stock change factors as well as OC flow dynamics
within the agricultural system.

This is critically important since managing agricultural soil might be
one of the few setscrew for humans to \emph{naturally} combat climate
change (ref. to natural climate solutions). Other \emph{natural}
options, which might be summarized by restoring potential natural stocks
via renaturation of managed land, seem to be often limited due to
increased pressure on the land systems for feed, food and energy demand
of a growing and developing population on earth.

\newpage

\section{Method (50)}

\hypertarget{sec:carbonbudget}{%
\subsection{Carbon Stocks following (new) Tier 2 method
(50)}\label{sec:carbonbudget}}

Following the tier 2 approach of the refinement of IPCC guidelines
vol.~4 (\citet{ipcc_2019_2019}), we estimate global land-use type
specific soil organic carbon (SOC) stocks for cropland and natural
vegetation on half-degree resolution from 1965 to 2010. We assume the
actual SOC state converges towards a stable steady state, that itself is
changing over time and space depending on biophysical, climatic and
agronomic conditions. Therefore we conduct the following three steps
within each yearly timestep: (1) We calculate annual land-use (sub-)type
specific steady states and decay rates for SOC stocks, (2) We account
for land conversion by transferring SOC between land-use types and (3)
We estimate SOC stocks based on the stocks of the previous time period,
the steady state stocks and the decay rate.

\subsubsection{Steady-state SOC stocks and decay rates}

In a simple first order kinetic approach the steady-state soil organic
carbon stocks \(SOC^{eq}\) are given by \begin{equation}
SOC^{eq}_t =\frac{C^{\textrm{in}}_t}{k_t}
\label{eq:inoutflow}
\end{equation} with \(C^{\textrm{in}}\) being carbon inputs to the soil
and \(k\) denoting the soil organic carbon decay rate. We use for our
calculations the steady-state method of the refinement of the IPCC
guidelines vol.~4 (\citet{ipcc_2019_2019}) for mineral soils, which
assume three soil carbon sub-pools (active, slow and passive) and
entangled dynamics between them. Annual carbon inflow to each sub-pool
(see @ref(sec:carboninputs)) and annual decay rates (see
@ref(sec:tier2)) of each sub-pool are still the key components to
determining steady-state SOC stocks.

BB: I would include the t, also to show that the steady-state is
time-dependent.

\hypertarget{sec:carboninputs}{%
\paragraph{Carbon Inputs to the Soil}\label{sec:carboninputs}}

We account for different carbon input sources depending on the land-use
type (see table @ref(tab:datasourceinputs)). Name here the inputs, and
refer to the respective sections, dont just refer to the table.
Following the IPCC methodology carbon inputs are disaggregated into
metabolic and structural components depending on their lignin and
nitrogen content (see @ref(ipcc\_2019\_2019)). For each component the
sum over all carbon input sources is allocated to the respective SOC
sub-pools via transfer coefficients. This implies that not only the
amount of carbon, but also their structural composition is determining
the effective inflow. Data sources for all considered carbon inputs as
well as for lignin and nitrogen content can be found in table
@ref(tab:datasourceinputs).

 \begin{table*}[h]
 \caption{Type and data sources for carbon inputs to different land-use types }
 \begin{tabular}{l l l l}
 \tophline
  land-use types BB: table header should be bold   & source of carbon inputs & data source & nitrogen and lignin content \\
 \middlehline
 \multirow{3}{*}{Cropland} & residues & FAOSTAT, LPJmL4 [2, sec:residues] & default values given by [2]  \\
                            & dead below ground biomass of crops & FAOSTAT, LPJmL4 [2, sec:residues] & default values given by [2] \\
                            & manure & FAOSTAT, Isabelle [2, sec:manure] & default values given by [2] \\
                            \hline
  Natural vegetation        & annual litterfall & LPJmL4 [4]& \begin{minipage}[t]{0.28\columnwidth}\raggedright\strut Nitrogen and lignin content of tree compartments used in CENTURY [4] \strut \end{minipage}\tabularnewline
 \bottomhline
 \end{tabular}
 \label{tab:datasourceinputs}
 \belowtable{}
 \end{table*}

\hypertarget{sec:tier2}{%
\paragraph{Soil Organic Carbon decay (300)}\label{sec:tier2}}

The sub-pool specific decay rates are influenced by climatic conditions,
biophysical and biochemical soil properties as well as management
factors that all vary over time (t) and space (i). Following the
steady-state method of the refinement of the IPCC guidelines vol.~4
(\citet{ipcc_2019_2019}) for mineral soils we consider temperature
(temp), water (wat), sand fraction (sf) and tillage (till) effects to
account for spatial variation of decay rates. Thus \(k_{sub}\) is given
by

\begin{equation}
\begin{aligned}
& k_{active,t,i}  & = &~ k_{active}  ~ &\cdot~ temp_{t,i} ~ &\cdot~ wat_{t,i} ~ &\cdot~ till_{t,i} ~ & \cdot~ sf_{t,i}\\
& k_{slow,t,i}    & = &~ k_{slow}    ~ &\cdot~ temp_{t,i} ~ &\cdot~ wat_{t,i} ~ &\cdot~ till_{t,i} ~ &\\
& k_{passive,t,i} & = &~ k_{passive} ~ &\cdot~ temp_{t,i} ~ &\cdot~ wat_{t,i} ~ & ~ &
\label{eq:decayrates}
\end{aligned}
\end{equation}

For cropland we distinguish the effect of different tillage (see
@ref(\#sec:tillage)) and irrigation (see @ref(\#sec:irrigation))
practices on decay rates, whereas on natural vegetation, we assume
rainfed and non-tilled conditions. Data sources as well as considered
effects for each land-use types are shown in table
@ref(tab:datasourcedecay). To account for variations of decay rates
within each grid cell due to different tillage and irrigation regimes,
average rates based on area shares are calculated.

 \begin{table*}[h]
 \caption{Type and data sources for carbon inputs to different land-use types}
 \begin{tabular}{l l l l}
 \tophline
  land-use types   & type of decay driver & parameter use to represent driver & data source \\
 \middlehline
 \multirow{2}{*}{all} & Soil quality & Sand fraction of the first 0-30 cm &  [SoilGrids]  \\
                      \cline{2-4}
                      
                      & Mircobial activity & air temperature & [CRUp4.0] \\
                      \cline{2-4}
                      
                      & Water restriction & precipitation \& potential evapotranspiration & [CRUp4.0] \\
                      \cline{1-4}
\multirow{2}{*}{\begin{minipage}[t]{0.2\columnwidth}\raggedright\strut Cropland\\(additionally)\strut\end{minipage}} & Water restriction*  & irrigation  & [sec:irrigation] \\ 
                      \cline{2-4}
                      
                      & Soil disturbance & tillage & [sec:tillage] \\
 \bottomhline
 \end{tabular}
 \label{tab:datasourcedecay}
 \belowtable{}
 \end{table*}

\subsubsection{SOC transfer between land-use types}

We calculate SOC stocks based on the area shares of land-use types (lut)
within the half-degree grid cells (i). If land is converted from one
land-use type into others (!lut), the respective share of the SOC stocks
is reallocated. We account for land conversion at the beginning of each
time step \(t\) by calculating a preliminary stock \(SOC_{lut,t*}\) via

\begin{equation}
SOC_{lut,t*} = SOC_{lut,t-1} - \frac{SOC_{lut,t-1}}{A_{lut,t-1}} \cdot  AR_{lut,t} + \frac{SOC_{!lut,t-1}}{A_{!lut,t-1}} \cdot  AE_{lut,t} \qquad \forall sub, i  
\label{eq:ctransfer}
\end{equation}

with \(A\) being the area, \(AR\) the area reduction and \(AE\) the area
expansion for a given land-use type \(lut\). Note that \(!lut\) denotes
the sum over all other land-use types, which decreases in the specific
time step \(t\). Data sources and methodology on land-use states and
changes are described in @ref(sec:landuse).

\subsubsection{Total SOC stocks}

Carbon stocks \(SOC\) for each sub-pool (sub) converge towards the
calculated steady-state stock \(SOC^{eq}\) for each land-use types
(lut), each sub-pool (sub) and each annual time step (t) as represented
in equation @ref(eq:steadystate).

\begin{equation}
SOC_{t} = SOC_{t-1} + (SOC^{eq}_{t} - SOC_{t-1}) \cdot k_{t} \cdot 1\unit{a} \qquad \forall lut, sub, i.
\label{eq:steadystate}
\end{equation}

This equation can also be reformulated to a massbalance equation as
follows:

\begin{equation}
SOC_{t} = SOC_{t-1} - k_{t} \cdot 1\unit{a} + C^{\textrm{in}}  \qquad \forall lut, sub, i.
\label{eq:steadystate}
\end{equation}

The global SOC stock for each time step can than be calculated via

\begin{equation}
SOC_{t} = \sum_{i} \underbrace{\sum_{lut} \overbrace{\sum_{sub} SOC_{lut, sub, t, i}.}^{\text{$SOC_{lut, t, i}$ - land-use type specific SOC stock within cell}}}_{\text{$SOC_{t, i}$ - total SOC stock within cell}}
\label{eq:totalstock}
\end{equation}

\subsubsection{Initialisation of SOC pools}

To initialize all SOC sub-pools we assume that cropped land natural
vegetation * steady-states or * spin up

\newpage

\hypertarget{sec:tier1}{%
\subsection{Carbon Budget following Tier 1 (150)}\label{sec:tier1}}

Additionally to the tier 2 approach of the refinement of IPCC guidelines
vol.~4 (\citet{ipcc_2019_2019}), we also estimate SOC pools using the
IPCC tier 1 approach of IPCC guidelines vol.~4 (\citet{ipcc_2006_2006})
for comparison. Here, stocks are estimated via stock change factors
given by the IPCC for the topsoil (0-30 cm) and based on a review of
measurement data. The factors differentiate different crop and
management systems reflecting different dynamics under changed in- and
outflows without explicitly tracking these. The SOC stocks as thus
calulated

\begin{equation}
SOC_{\text{target}} = \sum_{c,s,i} SOC_{\text{ref}_{c,s,i}} \cdot F_{\text{LU}_{c,s,i}} \cdot F_{\text{MG}_{c,s,i}} \cdot F_{\text{I}_{c,s,i}} \cdot A_{c,s,i}
\label{eq:tier1}
\end{equation}

\textless!- also include an equation here --\textgreater{} \textless!-
even if there are just ``copied'' out of te guidelines so to say?
--\textgreater{} \textless!- more details will follow - how deep to go?
--\textgreater{}

\hypertarget{sec:agrimanagement}{%
\subsection{Agricultural management data on 0.5 degrees
(50)}\label{sec:agrimanagement}}

Agricultural management data is estimated using the R library moinput
({[}cite git link{]}), which compiles country-specific FAO production
and cropland statistics (\citep{FAOSTAT}) to a comprehensive and
constistent data suite. BB: Isnt the whole calculation based on R
libraries? The data is prepared in 5 year time steps from 1965 to 2010,
which also restricts our analysis to this time span. For all the
following data, if not declared differently, we interpolate values
linearly between the time steps and hold it constant before the first
time step from 1961 to 1965. BB: too short for a spin-up, so dont use
this word. Maybe also just ignore this detail.

\hypertarget{sec:landuse}{%
\subsubsection{Landuse and Landuse Change (150)}\label{sec:landuse}}

Land-use patterns are based on the Land-Use Harmonization 2 (LUH2,
\citep{LUH2}) data set, which we aggregate from quarter degree to half
degree resolution. We disaggregate the five different cropland
subcategories (c3ann, c3per, c4ann, c4per, c3nfx) of LUH2 into our 17
crop groups, applying the relative shares for each gridcell based on the
country and year specific area harvested shares of FAOSTAT data
(\citep{FAOSTAT}) (see @ref(append:Table\_luh2fao2mag) for more details
on the crop type mapping). Land-use transitions are calculated as net
area differences of the land-use data on half-degree.

\subsubsection{Crop, Crop Residues and Pasture Production (300)}

\paragraph{Crop Production}

Using half-degree yield data from LPJmL (\citep{LPJmL4_1}) as well as
half-degree cropland patterns (see @ref(\#sec:landuse)) we compile crop
group specific half-degree production patterns. We calibrate cellular
yields with one country-level calibration factor for each crop group to
meet historical FAOSTAT production (\citep{FAOSTAT}). Note that by using
physical cropland areas we account for multiple crop harvest events as
well as for fallows.

\paragraph{Crop Residue Production}

Crop residue production and management is based on a revised methodology
of (\citep{bodirsky2012}) and will be explained in key aspects again due
to its central role for soil carbon modelling. Starting from crop
production estimates of the harvested organs and their respective crop
area, we estimate above-ground (ag) and below-ground (bg) residual
biomass using yield-dependent harvest indices and shoot:root ratios. We
assume that all bg residues are recycled to the soil, whereas ag
residues can be burned or harvested for other purposes such as feeding
animals (\citep{weindl}), fuel or for material use. BB: One new thing is
here that we use a doublecropping factor to account for lower harvest
indices of multicropping vs a very high single-cropped yield

\subparagraph{Burned Residues}

A fixed share of the ag residues is assumed to be burned on field
depending on the per-capita income of the country. Following
\citep{smil1999}) we assume 25\% burn share for low-income countries
according to worldbank definitions (\(<\,1000\,\tfrac{USD}{yr}\)), 15\%
for high-income (\(>\,10000\,\tfrac{USD}{yr}\) and linearly interpolate
shares for all middle-income countries depending on their per-capita
income. Depending on the crop type 80--90\% of the residue carbon burned
on the fields are lost within the combustion process
(\citep{ipcc_2006_2006}).

\subparagraph{Residue Usage}

We compile out of our 17 crop groups, three used residue groups (straw,
high-lignin and low-lignin residues) with additional demand for other
purposes and one residues with no double use (see
@ref(append:Table\_kcr2kres)).

Residue feed demand for five different livestock groups is based on
country- and residue-group-specific feed basekts (see \citep{weindl})
taking available ag residual biomass as well as livestock productivity
into account.

We estimate a material-use share for the straw residues group of 5\% and
a fuel-share of 10\% for all used residues groups in low income
countries according to worldbank definitions
(\(<\,1000\,\tfrac{USD}{yr}\)). For high-income
(\(>\,10000\,\tfrac{USD}{yr}\) no withdrawl for material or fuel use is
assumend, leaving middle-income countries with linearly interpolate
shares depending on their per-capita income.

The remaining ag residues as well as all bg residues are assumend to be
recycled to the soil. We cut high recycling shares per hectar at the
95\%-percentile to corrected for outliers.

\paragraph{Pasture Production}

Using livestock production statistics as well as feed mix assumptions as
describted in (\citep{weindl}) we estimating country specific pasture
production. Following the same approach as for crop production we
disaggregate and calibrate half-degree pasture production pattern from
grass yields from LPJmL and pasture area and rangeland patterns ( (see
@ref(\#sec:landuse))) to derive half-degree pasture production patterns.
BB: Oh, does that mean we assume that only the pasture grows that is
being eaten? Thats not too much, is it? Maybe ask Isabelle for a better
solution (e.g.~direct pasture yields from LPJmL). And then the ``pasture
production'' is just the part being removed by grazing. Does the IPCC
specifiy something in respect to pastures?

\paragraph{Dry Matter to Carbon Transformation}

To transform dry matter estimates into carbon, we compiled crop-group
and plant part specific carbon to dry matter (c:dm) ratios (see
@ref(append:Table\_c2dm)) (\citet{Ma2018}).

\subsubsection{Livestock Distribution and Manure Excretion (300)}

\paragraph{Livestock Distribution}

To disaggregate country level FAOSTAT livestock production values to
half-degree pattern, we use the following rule based assumptions which
were inspired by the approach of \citep{gilbert} and uses feed basket
assumptions based on a revised methodology of \citet{weindl}. We
differentiate ruminant and monogastric systems, as well as extensive and
an intensive systems. For ruminants, we use the country-level shares of
gras within the feed baskets to split up pasture-fed, extensive systems
from the rather intensive crop-fed livestock. For extensive dairy and
ruminant meat production we estimate, that livestock is located in
proximity to pastures and rangelands and distribute manure excretion
proportional to pasture production of all half-degree gridcell of a
country. Intensive dairy and ruminant meat production is assumend to be
located proportinal to crop production to have short transport distances
for feed stuff.

BB: What does not become clear to me: Do the extensive cows only eat
pasture, and the intensive one only crops? Does that in the end mean we
just redistirbute grazed pasture to pastures and crop-based feed to
croplands proportional to production? Or is there more to it?

For poultry, egg and monogastric meat production we use the per-capita
income of the country to divide into intensive and extensive production
systems. For low-income countries according to worldbank definitions
(\textless1000 USD/yr), we assume extensive production systems. We
locate them according to built-up areas shares based on the idea that
these animals are held in households, subsistence or small-holder
farming systems with a high labour per animal ratio. Intensive
production is distributed within a country using the crop production
share, assuming that feed availability is the most driving factor for
livestock location.

\paragraph{Manure Excretion, Storage and Recycling}

Manure production and management is based on a revised methodology of
(\citep{bodirsky2012}) and will be explained in key aspects again due to
its central role for soil carbon modelling. Based on the gridded
livestock distribution we calculate excretions by estimating the
nitrogen balance of the livestock system on the basis of comprehensive
livestock feed baskets (\citep{weindl}), assuming that all nitrogen in
protein feed intake, minus the nitrogen in the slaughter mass, is
excreted. Carbon in excreted manure is estimated by applying fixed C:N
ratios (given by \citep[(][]{ipcc_2019_2019}). Depending on the feed
system we assume manure to be handled in four different ways: All manure
orginated from pasture feed intake is excreted directly to pastures and
rangelands (pasture grazing), deducting manure collected as fuel. Manure
fuel shares are estimated using IPCC default values
(\citet{ippc_2006_2006}). Whereas for low-income countries according to
worldbank definitions (\textless1000 USD/yr), we adopt a share of 25\%
of crop residues in feed intake directly consumend and excreted on crop
fields (stubble grazing), we do not consider any stubble grazing in
high-income countries (\(>\,10000\,\tfrac{USD}{yr}\), leaving
middle-income countries with linearly interpolate shares depending on
their per-capita income. For all other feed items we assume the manure
to be stored in animal waste management systems associated to animal
houses. To estimate the carbon actually recycled to the soil, we account
for carbon losses during storage and recycling shares in different
animal waste management and grazing systems. Whereas we assume no losses
for pasture and stubble grazing, we consider that the manure collected
as fuel is not recycled. For manure stored in different animal waste
management system we compiled carbon loss rates partly depending on the
nitrogen loss rates as specified in \citep{bodirsky2012} (see
@ref(append:Table\_clossAWMS))).

\subsubsection{Irrigation (100)}

BB: Where does the irrigated area come from? Simple growing period
calculations together with irrigation shares of LUH2v2 are used to
estimate water effects on decay rates.

\subsubsection{Tillage (100)}

Tillage data sets of {[}Vera, others{]} together with rules are used to
drive tillage effect on decay rates. \newpage

\section{Results}

\begin{figure}
\includegraphics[width=17.78in]{../OldDataNotTracked/images/maps/CShare_1965} \includegraphics[width=17.78in]{../OldDataNotTracked/images/maps/CIncrease_1965} \caption{Test}\label{fig:unnamed-chunk-6}
\end{figure}

\begin{figure}
\includegraphics[width=12cm]{../OldDataNotTracked/images/figs_draft-1} \caption{two column figure}\label{fig:unnamed-chunk-7}
\end{figure}

\begin{figure}
\includegraphics[width=12cm]{../OldDataNotTracked/images/figs_draft-2} \caption{two column figure}\label{fig:unnamed-chunk-8}
\end{figure}

\begin{figure}
\includegraphics[width=12cm]{../OldDataNotTracked/images/figs_draft-3} \caption{two column figure}\label{fig:unnamed-chunk-9}
\end{figure}

\begin{figure}
\includegraphics[width=12cm]{../OldDataNotTracked/images/RegionPlot+Valid} \caption{two column figure}\label{fig:unnamed-chunk-10}
\end{figure}

\begin{figure}
\includegraphics[width=12cm]{../OldDataNotTracked/images/maps/CShare_4ColorClimates} \caption{one column figure}\label{fig:unnamed-chunk-11}
\end{figure}

\begin{Shaded}
\begin{Highlighting}[]
\KeywordTok{par}\NormalTok{(}\DataTypeTok{mar =} \KeywordTok{c}\NormalTok{(}\DecValTok{4}\NormalTok{, }\DecValTok{4}\NormalTok{, }\FloatTok{0.1}\NormalTok{, }\FloatTok{0.1}\NormalTok{))}
\end{Highlighting}
\end{Shaded}

\begin{table}

\caption{\label{tab:table}SCF compared to potential natural state}
\centering
\begin{tabular}[t]{l|r|r|r|r}
\hline
  & tropical\_moist & tropical\_dry & temperate\_dry & temperate\_moist\\
\hline
1965 & 0.25 & 0.33 & 0.55 & 0.48\\
\hline
1970 & 0.29 & 0.36 & 0.56 & 0.50\\
\hline
1975 & 0.32 & 0.38 & 0.57 & 0.51\\
\hline
1980 & 0.34 & 0.38 & 0.56 & 0.53\\
\hline
1985 & 0.36 & 0.40 & 0.60 & 0.56\\
\hline
1990 & 0.38 & 0.42 & 0.63 & 0.58\\
\hline
1995 & 0.40 & 0.45 & 0.63 & 0.59\\
\hline
2000 & 0.42 & 0.47 & 0.64 & 0.60\\
\hline
2005 & 0.45 & 0.54 & 0.69 & 0.65\\
\hline
2010 & 0.50 & 0.58 & 0.73 & 0.67\\
\hline
\end{tabular}
\end{table}

\begin{table}

\caption{\label{tab:table}SCF compared to actual natural state}
\centering
\begin{tabular}[t]{l|r|r|r|r}
\hline
  & tropical\_moist & tropical\_dry & temperate\_dry & temperate\_moist\\
\hline
1965 & 0.25 & 0.33 & 0.55 & 0.48\\
\hline
1970 & 0.29 & 0.36 & 0.56 & 0.50\\
\hline
1975 & 0.32 & 0.38 & 0.57 & 0.51\\
\hline
1980 & 0.34 & 0.38 & 0.56 & 0.53\\
\hline
1985 & 0.36 & 0.40 & 0.60 & 0.56\\
\hline
1990 & 0.38 & 0.42 & 0.63 & 0.58\\
\hline
1995 & 0.40 & 0.45 & 0.63 & 0.59\\
\hline
2000 & 0.42 & 0.47 & 0.64 & 0.60\\
\hline
2005 & 0.45 & 0.54 & 0.69 & 0.65\\
\hline
2010 & 0.50 & 0.58 & 0.74 & 0.68\\
\hline
\end{tabular}
\end{table}

\begin{table}

\caption{\label{tab:table}Steady state SCF compared to actual natural state}
\centering
\begin{tabular}[t]{l|r|r|r|r}
\hline
  & tropical\_moist & tropical\_dry & temperate\_dry & temperate\_moist\\
\hline
1965 & 0.22 & 0.32 & 0.53 & 0.47\\
\hline
1970 & 0.24 & 0.46 & 0.52 & 0.53\\
\hline
1975 & 0.25 & 0.51 & 0.69 & 0.55\\
\hline
1980 & 0.26 & 0.45 & 0.68 & 0.63\\
\hline
1985 & 0.29 & 0.57 & 0.90 & 0.87\\
\hline
1990 & 0.31 & 0.60 & 0.90 & 0.87\\
\hline
1995 & 0.35 & 0.62 & 0.85 & 0.72\\
\hline
2000 & 0.37 & 0.67 & 0.85 & 0.78\\
\hline
2005 & 0.41 & 0.78 & 1.05 & 0.91\\
\hline
2010 & 0.47 & 0.83 & 1.30 & 0.95\\
\hline
\end{tabular}
\end{table}
\newpage

\section{Discussion}

Big points:

\begin{itemize}
\item
  SOC initialization is bad (Task: find a better representation)
\item
  natural land representation is lacking proper parametrization of
  natural input (n, lg content of litterfall) (--\textgreater{} ask
  Christop again)
\item
  boreal zone and dry regions are not well represented (maybe because of
  the bad parameterization of the soil model for natural soils)
\item
  As pointed out by Keel (2017), Smith (2019) the results might be
  highly sensitive to carbon input calculations more precisly to below
  and above ground residue carbon estimates derived from harvested
  quantities. It has been questioned, that below ground residues might
  increase with a fixed root:shot ratio (maybe specifically in high end
  farming systems (?)). Following this argumentation SOC results shown
  in this study might especially in high-yielding farming system (europe
  etc.) overestimate actual SOC stocks.
\item
  fertilizer interaction (--\textgreater{} ask LPJmLer again)
\item
  diaggregation of manure with urban area is leeding to a lot of
  displaced manure (?) that is cut off
\item
  mismatches between FAO/LUH
\end{itemize}

Shortcommings:

\begin{itemize}
\item
  Carbon displacement via leaching and erosion is neglected in this
  study.
\item
  Non-net/Gross land use transitions are not tracked in this study.
\item
  Within cropland we do not track area transitions, but rather look at
  statistical distributions of the crop functional types. Due to crop
  rotations and missing data on crop specific distributions, these
  transitions would be any way rather uncertain. \newpage
\end{itemize}

\conclusions

The conclusion goes here. You can modify the section name with
\texttt{\textbackslash{}conclusions{[}modified\ heading\ if\ necessary{]}}.
\newpage




\codedataavailability{use this to add a statement when having data sets
and software code
available} %% use this section when having data sets and software code available



%%%%%%%%%%%%%%%%%%%%%%%%%%%%%%%%%%%%%%%%%%
%% optional

%%%%%%%%%%%%%%%%%%%%%%%%%%%%%%%%%%%%%%%%%%
\appendix
\section{Figures and tables in appendices}
\subsection{Option 1}

If you sorted all figures and tables into the sections of the text,
please also sort the appendix figures and appendix tables into the
respective appendix sections. They will be correctly named
automatically.

\subsection{Option 2}

If you put all figures after the reference list, please insert appendix
tables and figures after the normal tables and figures.

\texttt{\textbackslash{}appendixfigures} needs to be added in front of
appendix figures \texttt{\textbackslash{}appendixtables} needs to be
added in front of appendix tables

Please add \texttt{\textbackslash{}clearpage} between each table and/or
figure. Further guidelines on figures and tables can be found below.
Regarding figures and tables in appendices, the following two options
are possible depending on your general handling of figures and tables in
the manuscript environment: To rename them correctly to A1, A2, etc.,
please add the following commands in front of them:
\noappendix

%%%%%%%%%%%%%%%%%%%%%%%%%%%%%%%%%%%%%%%%%%
\authorcontribution{Karstens wrote code and paper build on work of
Bodirsky. Bodirsky and Popp revised paper.} %% optional section

%%%%%%%%%%%%%%%%%%%%%%%%%%%%%%%%%%%%%%%%%%
\competinginterests{The authors declare no competing
interests.} %% this section is mandatory even if you declare that no competing interests are present

%%%%%%%%%%%%%%%%%%%%%%%%%%%%%%%%%%%%%%%%%%
\disclaimer{We like Copernicus.} %% optional section

%%%%%%%%%%%%%%%%%%%%%%%%%%%%%%%%%%%%%%%%%%
\begin{acknowledgements}
Thanks to the rticles contributors!
\end{acknowledgements}

%% REFERENCES
%% DN: pre-configured to BibTeX for rticles

%% The reference list is compiled as follows:
%%
%% \begin{thebibliography}{}
%%
%% \bibitem[AUTHOR(YEAR)]{LABEL1}
%% REFERENCE 1
%%
%% \bibitem[AUTHOR(YEAR)]{LABEL2}
%% REFERENCE 2
%%
%% \end{thebibliography}

%% Since the Copernicus LaTeX package includes the BibTeX style file copernicus.bst,
%% authors experienced with BibTeX only have to include the following two lines:
%%
\bibliographystyle{copernicus}
\bibliography{SOCbudget.bib}
%%
%% URLs and DOIs can be entered in your BibTeX file as:
%%
%% URL = {http://www.xyz.org/~jones/idx_g.htm}
%% DOI = {10.5194/xyz}


%% LITERATURE CITATIONS
%%
%% command                        & example result
%% \citet{jones90}|               & Jones et al. (1990)
%% \citep{jones90}|               & (Jones et al., 1990)
%% \citep{jones90,jones93}|       & (Jones et al., 1990, 1993)
%% \citep[p.~32]{jones90}|        & (Jones et al., 1990, p.~32)
%% \citep[e.g.,][]{jones90}|      & (e.g., Jones et al., 1990)
%% \citep[e.g.,][p.~32]{jones90}| & (e.g., Jones et al., 1990, p.~32)
%% \citeauthor{jones90}|          & Jones et al.
%% \citeyear{jones90}|            & 1990

\end{document}
